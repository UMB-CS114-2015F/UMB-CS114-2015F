% -----------------------------------------------------------------------------
% The MIT License (MIT)
%
% Copyright (c) 2015 Pejman Ghorbanzade
%
% Permission is hereby granted, free of charge, to any person obtaining a copy
% of this software and associated documentation files (the "Software"), to deal
% in the Software without restriction, including without limitation the rights
% to use, copy, modify, merge, publish, distribute, sublicense, and/or sell
% copies of the Software, and to permit persons to whom the Software is
% furnished to do so, subject to the following conditions:
%
% The above copyright notice and this permission notice shall be included in
% all copies or substantial portions of the Software.
%
% THE SOFTWARE IS PROVIDED "AS IS", WITHOUT WARRANTY OF ANY KIND, EXPRESS OR
% IMPLIED, INCLUDING BUT NOT LIMITED TO THE WARRANTIES OF MERCHANTABILITY,
% FITNESS FOR A PARTICULAR PURPOSE AND NONINFRINGEMENT. IN NO EVENT SHALL THE
% AUTHORS OR COPYRIGHT HOLDERS BE LIABLE FOR ANY CLAIM, DAMAGES OR OTHER
% LIABILITY, WHETHER IN AN ACTION OF CONTRACT, TORT OR OTHERWISE, ARISING FROM,
% OUT OF OR IN CONNECTION WITH THE SOFTWARE OR THE USE OR OTHER DEALINGS IN
% THE SOFTWARE.
% -----------------------------------------------------------------------------

\section*{Question 1}
The following code snippet does not compile. Rewrite the code snippet and create a program \texttt{HelloUMass.java} by resolving problems of the given code such that it compiles successfully and prints \textit{``Hello World!"} on the console once executed.

\lstset{language=Java}
\begin{lstlisting}
public class HelloUMass
	public static void main(String args) {
		System.out.pintln(Hello UMass Boston!")
	}
}
\end{lstlisting}

\subsection*{Solution}
\lstset{language=Java,tabsize=2}
\begin{lstlisting}
public class HelloUMass {
	public static void main(String[] args) {
		System.out.println("Hello UMass Boston!");
	}
}
\end{lstlisting}

\section*{Question 2}
Write a program \texttt{Weather.java} that takes two command-line arguments as \textit{city} and \textit{condition} and functions as shown in following examples.

\begin{terminal}
$ java Weather Boston sunny
Boston is sunny today!
$ java Weather Cambridge cloudy
Cambridge is cloudy today!
\end{terminal}

\subsection*{Solution}
\lstset{language=Java,tabsize=2}
\begin{lstlisting}
public class Weather {
	public static void main(String[] args) {
		System.out.println(args[0] + " is " + args[1] + " today!");
	}
}
\end{lstlisting}

\section*{Question 3}
Write a program \texttt{Temperature.java} that takes temperature in Celsius as command-line argument and prints it's equivalent in fahrenheit. Your program is expected to function as shown in the following example.

\begin{terminal}
$ javac TemperatureCalculator
$ java TemperatureCalculator 12
12.0 Celsius is 53.6 fahrenheit.
\end{terminal}

\subsection*{Solution}
\lstset{language=Java,tabsize=2}
\begin{lstlisting}
public class Temperature {
	public static void main(String[] args) {
		double celsius = Double.parseDouble(args[0]);
		double fahrenheit = celsius * 1.8 + 32;
		System.out.println(celsius + `` Celsius is " + fahrenheit + `` fahrenheit.");
	}
}
\end{lstlisting}

\section*{Question 4}
Given an airplane's acceleration \texttt{a} and take-off speed \texttt{v}, the minimum runway length needed for the airplane to take off can be computed using the formula given in Equation \ref{eq1}.

\begin{equation}
\textrm{length} = \frac{v^2}{2a}
\label{eq1}
\end{equation}

Write a program \texttt{Runway.java} that takes parameters \texttt{v} in meters/second and \texttt{a} in meters/second-squared respectively as command-line arguments and displays the minimum runway length required for it. An example to illustrate the output format is given below.

\begin{terminal}
$ javac Runway.java
$ java Runway 60 3.5
Speed of the airplane is assumed 60.000 m/s
Acceleration of the airplane is assumed 3.500 m/s2
Minimum runway length is 514.286 m
\end{terminal}

\subsection*{Solution}
\lstset{language=Java,tabsize=2}
\begin{lstlisting}
public class Runway {
	public static void main(String[] args) {
		double speed = Double.parseDouble(args[0]);
		double acceleration = Double.parseDouble(args[1]);
		double runwayLength = Math.pow(speed, 2) / (2 * acceleration);
		System.out.printf("Speed of the airplane is assumed %.3f m/s\n", speed);
		System.out.printf("Acceleration of the airplane is assumed %.3f m/s2\n", acceleration);
		System.out.printf("Minimum runway length is %.3f m\n", runwayLength);
	}
}
\end{lstlisting}

\section*{Question 5}
Due to the combined effects of gravitation and rotation, the Earth's shape is roughly that of a sphere slightly flattened in the direction of its axis. Consiquently, the Earth is often approximated by an oblate spheroid instead of a sphere. Mathematically, a spheroid is a quadric surface obtained by rotating an ellipse about one of its principal axes.

Equation \ref{eq2} describes a spheroid centred at the origin with its semi-axes aligned along the coordinate axes, in which semi-axis \texttt{a} is equatorial radius of the spheroid and \texttt{c} is the distance from centre to pole along the symmetry axis.

\begin{equation}
\frac{x^2+y^2}{a^2} + \frac{z^2}{c^2} = 1
\label{eq2}
\end{equation}

Write a program \texttt{Spheroid.java} that takes respectively parameters \texttt{a} and \texttt{c} as command-line arguments and calculates surface area of an oblate spheroid using Equation \ref{eq3}.

\begin{equation}
S_{\text{oblate}} = 2\pi a^2 + \pi \frac{c^2}{e} \ln \left(\frac{1+e}{1-e} \right)
\label{eq3}
\end{equation}

where variable \texttt{e} is as given in Equation \ref{eq4}

\begin{equation}
e = \sqrt{1 - \frac{c^2}{a^2}}
\label{eq4}
\end{equation}

The example given below illustrates the accepted output format.

\begin{terminal}
$ javac Spheroid.java
$ java Spheroid 6 5
Surface area is 403.050
\end{terminal}

\subsection*{Solution}
\lstset{language=Java,tabsize=2}
\begin{lstlisting}
public class Spheroid {
	public static void main(String[] args) {
		double a = Double.parseDouble(args[0]);
		double c = Double.parseDouble(args[1]);
		double e = Math.sqrt(1 - Math.pow(c, 2)/Math.pow(a, 2));
		double surface = 2 * Math.PI * Math.pow(a, 2) + Math.PI * Math.pow(c, 2) / e * Math.log( (1+e) / (1-e));
		System.out.printf("Surface area is %.3f\n", surface);
	}
}
\end{lstlisting}
