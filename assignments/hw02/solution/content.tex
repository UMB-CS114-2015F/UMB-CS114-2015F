% -----------------------------------------------------------------------------
% The MIT License (MIT)
%
% Copyright (c) 2015 Pejman Ghorbanzade
%
% Permission is hereby granted, free of charge, to any person obtaining a copy
% of this software and associated documentation files (the "Software"), to deal
% in the Software without restriction, including without limitation the rights
% to use, copy, modify, merge, publish, distribute, sublicense, and/or sell
% copies of the Software, and to permit persons to whom the Software is
% furnished to do so, subject to the following conditions:
%
% The above copyright notice and this permission notice shall be included in
% all copies or substantial portions of the Software.
%
% THE SOFTWARE IS PROVIDED "AS IS", WITHOUT WARRANTY OF ANY KIND, EXPRESS OR
% IMPLIED, INCLUDING BUT NOT LIMITED TO THE WARRANTIES OF MERCHANTABILITY,
% FITNESS FOR A PARTICULAR PURPOSE AND NONINFRINGEMENT. IN NO EVENT SHALL THE
% AUTHORS OR COPYRIGHT HOLDERS BE LIABLE FOR ANY CLAIM, DAMAGES OR OTHER
% LIABILITY, WHETHER IN AN ACTION OF CONTRACT, TORT OR OTHERWISE, ARISING FROM,
% OUT OF OR IN CONNECTION WITH THE SOFTWARE OR THE USE OR OTHER DEALINGS IN
% THE SOFTWARE.
% -----------------------------------------------------------------------------

\section*{Question 1}
Given coordinates of a point and a sphere in tridimensional space, a point can either be inside, outside or on the sphere.

Write a program \texttt{PointAndSphere.java} that takes respectively coordinates of a point and center of a sphere as command line arguments and prompts user for radius of the sphere.
Your program would then determine the location of the point with respect to the sphere.
Following is an example of an accepted output format.

\begin{verbatim}
$ java PointAndSphere 1 1 1 0 0 0
Radius of the sphere: 1.7
The point is outside the sphere.
$ java PointAndSphere 1 1 1 0 0 0
Radius of the sphere 1.8
The point is inside the sphere.
\end{verbatim}

\subsection*{Solution}
\lstset{language=Java,tabsize=2}
\begin{lstlisting}
import java.util.Scanner;
public class PointAndSphere {
	public static void main(String[] args) {
		Scanner input = new Scanner(System.in);
		double pointX = Double.parseDouble(args[0]);
		double pointY = Double.parseDouble(args[1]);
		double pointZ = Double.parseDouble(args[2]);
		double sphereX = Double.parseDouble(args[3]);
		double sphereY = Double.parseDouble(args[4]);
		double sphereZ = Double.parseDouble(args[5]);
		System.out.printf("Radius of the Sphere: ");
		double radius = input.nextDouble();
		input.close();
		double distance = Math.sqrt(Math.pow(sphereZ - pointZ, 2) + Math.pow(sphereY - pointY, 2) + Math.pow(sphereX - pointX, 2));
		if (distance < radius)
			System.out.println("The point is inside the sphere.");
		else if (distance == radius)
			System.out.println("THe point is on the sphere.");
		else
			System.out.println("The point is outside the sphere.");
	}
}
\end{lstlisting}

\section*{Question 2}
Using Leibniz Formula, the number $\pi$ can be approximated using the series given in Equation \ref{eq1}.

\begin{equation}
\pi = 4 \left( 1 - \frac{1}{3} + \frac{1}{5} + \cdots + \frac{(-1)^{i+1}}{2i-1} \right)
\label{eq1}
\end{equation}

Write a program \texttt{PiProximator.java} that prompts user for a number $i$ and approximates the number $\pi$ using the first $i$ sentences of Equation \ref{eq1}.
Your program should also print the approximation error compared to the \texttt{Math.PI} value.
Following is the expected output of a sample run of your program.

\begin{verbatim}
$ java PiProximator
Number of Sentences: 100
Approximate Value: 3.131593
Approximation Error: 0.010000
\end{verbatim}

\subsection*{Solution}
\lstset{language=Java,tabsize=2}
\begin{lstlisting}
import java.util.Scanner;
public class PiProximator {
	public static void main(String[] args) {
		Scanner input = new Scanner(System.in);
		System.out.printf("Number of Sentences: ");
		int num = input.nextInt();
		input.close();
		double sum = 0;
		for (int i = 1; i <= num; i ++)
			sum += Math.pow(-1, i + 1) / (2 * i - 1);
		double pi = 4 * sum;
		System.out.printf("Approximate Value: %.6f\n", pi);
		System.out.printf("Approximation Error: %.6f\n", Math.abs(Math.PI - pi));
	}
}
\end{lstlisting}

\section*{Question 3}
Write a program \texttt{PrimePrinter.java} that prompts user for a number $x$ and prints all prime numbers less than or equal to the number $x$.
Your program should function as shown in the example given below.

\begin{verbatim}
$ java PrimePrinter
Enter Upper Limit: 29
2 3 5 7 11 13 17 19 23 29
\end{verbatim}

\subsection*{Solution}
\lstset{language=Java,tabsize=2}
\begin{lstlisting}
import java.util.Scanner;
public class PrimePrinter {
	public static void main(String[] args) {
		Scanner input = new Scanner(System.in);
		System.out.printf("Enter Upper Limit: ");
		int limit = input.nextInt();
		input.close();
		for (int num = 2; num <= limit; num++) {
			boolean isPrime = true;
			for (int i = 2; i < num; i++)
				if (num % i == 0)
					isPrime = false;
			if (isPrime == true)
				System.out.printf("%d ", num);
		}
		System.out.println();
	}
}
\end{lstlisting}

\section*{Question 4}
Write a program \texttt{MatrixFiller.java} that prompts user for a number $x$ between 1 to 9 and generates an $x \times x$ matrix with random elements from 1 to $x^2$.
Following is an expected sample run of your program.

\begin{verbatim}
$ java MatrixFiller
Size of Matrix: 5
 05 21 07 25 11
 17 02 17 01 07
 25 14 14 08 07
 02 05 01 14 02
 02 14 14 13 07
\end{verbatim}

\subsection*{Solution}
\lstset{language=Java,tabsize=2}
\begin{lstlisting}
import java.util.Scanner;
public class MatrixFiller {
	public static void main(String[] args) {
		Scanner input = new Scanner(System.in);
		System.out.printf("Size of Matrix: ");
		int matrixSize = input.nextInt();
		input.close();
		if (0 < matrixSize && matrixSize < 10)
			for (int i = 0; i < matrixSize; i++) {
				for (int j = 0; j < matrixSize; j++) {
					int num = (int) Math.floor(Math.random() * Math.pow(matrixSize, 2)) + 1;
					System.out.printf(" %02d", num);
				}
				System.out.println();
			}
	}
}
\end{lstlisting}

\section*{Question 5}
Write a program \texttt{Pyramid.java} that prompts user for a number between 1 to 10 and displays a pyramid of the given format.

\begin{verbatim}
$ java Pyramid
Number of Rows: 5
                   1
               1   2   1
           1   2   4   2   1
       1   2   4   8   4   2   1
   1   2   4   8  16   8   4   2   1
\end{verbatim}

\subsection*{Solution}
\lstset{language=Java,tabsize=2}
\begin{lstlisting}
import java.util.Scanner;
public class Pyramid {
	public static void main(String[] args) {
		Scanner input = new Scanner(System.in);
		System.out.printf("Number of Rows: ");
		int numRows = input.nextInt();
		input.close();
		for (int i = 0; i < numRows; i++) {
			int numElements = i;
			int numIndents = numRows - i - 1;
			for (int j = 0; j < numIndents; j++)
				System.out.printf("    ");
			for (int j = 0; j < i; j++)
				System.out.printf("%4.0f", Math.pow(2, j));
			for (int j = i; j >= 0; j--)
				System.out.printf("%4.0f", Math.pow(2, j));
			System.out.println();
		}
	}
}
\end{lstlisting}
