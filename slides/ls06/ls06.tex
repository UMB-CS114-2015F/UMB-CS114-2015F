% -----------------------------------------------------------------------------
% The MIT License (MIT)
%
% Copyright (c) 2015 Pejman Ghorbanzade
%
% Permission is hereby granted, free of charge, to any person obtaining a copy
% of this software and associated documentation files (the "Software"), to deal
% in the Software without restriction, including without limitation the rights
% to use, copy, modify, merge, publish, distribute, sublicense, and/or sell
% copies of the Software, and to permit persons to whom the Software is
% furnished to do so, subject to the following conditions:
%
% The above copyright notice and this permission notice shall be included in
% all copies or substantial portions of the Software.
%
% THE SOFTWARE IS PROVIDED "AS IS", WITHOUT WARRANTY OF ANY KIND, EXPRESS OR
% IMPLIED, INCLUDING BUT NOT LIMITED TO THE WARRANTIES OF MERCHANTABILITY,
% FITNESS FOR A PARTICULAR PURPOSE AND NONINFRINGEMENT. IN NO EVENT SHALL THE
% AUTHORS OR COPYRIGHT HOLDERS BE LIABLE FOR ANY CLAIM, DAMAGES OR OTHER
% LIABILITY, WHETHER IN AN ACTION OF CONTRACT, TORT OR OTHERWISE, ARISING FROM,
% OUT OF OR IN CONNECTION WITH THE SOFTWARE OR THE USE OR OTHER DEALINGS IN
% THE SOFTWARE.
% -----------------------------------------------------------------------------

\def \topDirectory {../..}

\documentclass[10pt, compress]{beamer}

\usepackage{\topDirectory/template/style/directives}
% -----------------------------------------------------------------------------
% The MIT License (MIT)
%
% Copyright (c) 2015 Pejman Ghorbanzade
%
% Permission is hereby granted, free of charge, to any person obtaining a copy
% of this software and associated documentation files (the "Software"), to deal
% in the Software without restriction, including without limitation the rights
% to use, copy, modify, merge, publish, distribute, sublicense, and/or sell
% copies of the Software, and to permit persons to whom the Software is
% furnished to do so, subject to the following conditions:
%
% The above copyright notice and this permission notice shall be included in
% all copies or substantial portions of the Software.
%
% THE SOFTWARE IS PROVIDED "AS IS", WITHOUT WARRANTY OF ANY KIND, EXPRESS OR
% IMPLIED, INCLUDING BUT NOT LIMITED TO THE WARRANTIES OF MERCHANTABILITY,
% FITNESS FOR A PARTICULAR PURPOSE AND NONINFRINGEMENT. IN NO EVENT SHALL THE
% AUTHORS OR COPYRIGHT HOLDERS BE LIABLE FOR ANY CLAIM, DAMAGES OR OTHER
% LIABILITY, WHETHER IN AN ACTION OF CONTRACT, TORT OR OTHERWISE, ARISING FROM,
% OUT OF OR IN CONNECTION WITH THE SOFTWARE OR THE USE OR OTHER DEALINGS IN
% THE SOFTWARE.
% -----------------------------------------------------------------------------

\course{id}{CS114}
\course{name}{Introduction to Java}
\course{venue}{Mon/Wed, 5:30 PM - 6:45 PM}
\course{semester}{Fall 2015}
\course{department}{Department of Computer Science}
\course{university}{University of Massachusetts Boston}

\instructor{name}{Pejman Ghorbanzade}
\instructor{title}{}
\instructor{position}{Student Instructor}
\instructor{email}{pejman@cs.umb.edu}
\instructor{phone}{617-287-6419}
\instructor{office}{S-3-124B}
\instructor{office-hours}{Mon/Wed 16:00-17:30}
\instructor{address}{University of Massachusetts Boston, 100 Morrissey Blvd., Boston, MA}

\usepackage{\topDirectory/template/style/beamerthemeUmassLecture}
\doc{number}{6}
%\setbeamertemplate{footline}[text line]{}

\begin{document}
\prepareCover

\section{Course Administration}

\begin{slide}
	\begin{itemize}
		\item[] Student Forum now available in \href{http://www.ghorbanzade.com}{Course Website}.
		\item[] 37/40 user accounts at \href{http://www.ghorbanzade.com}{Course Website} created so far.
	\end{itemize}
\end{slide}

\begin{slide}
	\begin{block}{Overview}
		\begin{itemize}
			\item[] Class Math
			\item[] Data Conversion
		\end{itemize}
	\end{block}
\end{slide}

\section{Class Math}

\begin{slide}
	Class Math provides methods to perform basic numeric operations, efficiently.
	\begin{block}{Fields}
		\begin{columns}
			\begin{column}{0.5\textwidth}
				\begin{itemize}
					\item[] Neperian Number
				\end{itemize}
			\end{column}
			\begin{column}{0.5\textwidth}
				\begin{itemize}
					\item[] Pi Number
				\end{itemize}
			\end{column}
		\end{columns}
	\end{block}
	\begin{block}{Methods}
		\begin{columns}
			\begin{column}{0.5\textwidth}
				\begin{itemize}
					\item[] abs(a)
					\item[] floor(a)
					\item[] ceil(a)
					\item[] round(a)
					\item[] sqrt(a)
					\item[] cbrt(a)
				\end{itemize}
			\end{column}
			\begin{column}{0.5\textwidth}
				\begin{itemize}
					\item[] sin(a)
					\item[] asin(a)
					\item[] sinh(a)
					\item[] cos(a)
					\item[] acos(a)
					\item[] cosh(a)
				\end{itemize}
			\end{column}
		\end{columns}
	\end{block}
\end{slide}

\begin{slide}
	\begin{block}{Methods (cont'd)}
		\begin{columns}
			\begin{column}{0.5\textwidth}
				\begin{itemize}
					\item[] exp(a)
					\item[] log(a)
					\item[] tan(a)
					\item[] atan(a)
					\item[] atan2(a)
					\item[] tanh(a)
				\end{itemize}
			\end{column}
			\begin{column}{0.5\textwidth}
				\begin{itemize}
					\item[] max(a, b)
					\item[] min(a, b)
					\item[] pow(a, b)
					\item[] random()
					\item[] toRadians(a)
					\item[] toDegrees(a)
				\end{itemize}
			\end{column}
		\end{columns}
	\end{block}
	\begin{block}{Usage}
		\begin{minted}[fontsize=\small,tabsize=8]{java}
			double randomNum = Math.random();
			int upperBound = Math.ceil(randomNum);
			int lowerBound = Math.floor(randomNum);
		\end{minted}
	\end{block}
\end{slide}

\begin{slide}
	\begin{block}{How random is Math.random()?}
		Method Math.random() is a little biased. We will discuss better alternatives later.
	\end{block}
	\begin{block}{Method Math.random()}
		\begin{minted}[fontsize=\small,tabsize=8]{java}
			double randomNum1 = Math.random();
		\end{minted}
		Output is always less than or equal to zero and less than one. What if we want a random number between $x$ and $y$?
		\begin{minted}[fontsize=\small,tabsize=8]{java}
			double randomNum2 = x + (y-x) * Math.random();
		\end{minted}
	\end{block}
\end{slide}

\section{Data Conversion}

\begin{slide}
	\begin{itemize}
		\item[] All variables should be declared only once.
		\item[] Variables of a declared type can be assigned values of that type.
	\end{itemize}
	\begin{block}{Converting Primitive Types to String}
		\begin{minted}[fontsize=\small,tabsize=8]{java}
			int a = 64;
			double b = 124.52;
			boolean c = false;
			String strA = String.valueOf(a);
			String strB = String.valueOf(b);
			String strC = String.valueOf(c);
		\end{minted}
	\end{block}
\end{slide}

\begin{slide}
	\begin{block}{Explicit Type Conversion}
		\begin{minted}[fontsize=\small,tabsize=8]{java}
			String strA = "64";
			String strB = "124.52";
			String strC = "false";
			int a = Integer.parseInt(strA);
			double b = Double.parseDouble(strB);
			boolean c = Boolean.parseBoolean(strC);
		\end{minted}
		Many methods perform a type conversion by receiving input of one type and giving output of another type.
	\end{block}
\end{slide}

\begin{slide}
	\begin{block}{Explicit Type Casting}
		\begin{minted}[fontsize=\small,tabsize=8]{java}
			double a = 124.52;
			int b = (int) a;
			int c = (int) a + b;
			int d = (int) (a + b);
		\end{minted}
		Type casting has higher precedance than other operations. Beware! Type casting may involve information loss.
	\end{block}
\end{slide}

\begin{slide}
	\begin{block}{Automatic Promotion}
		\begin{minted}[fontsize=\small,tabsize=8]{java}
			int a = 5;
			int b = -15;
			int c = 9;
			double discriminant = Math.pow(b,2) - 4 * a *c;
		\end{minted}
		Java automatically promotes data types where it assumes we have been light-headed. In above example, variables a, b and c will first be converted to type double and then other operations are performed. Automatic promotion is safe as it doesn't involve information loss.
	\end{block}
\end{slide}

\plain{}{Keep Calm\\and\\Practice}

\end{document}
