% -----------------------------------------------------------------------------
% The MIT License (MIT)
%
% Copyright (c) 2015 Pejman Ghorbanzade
%
% Permission is hereby granted, free of charge, to any person obtaining a copy
% of this software and associated documentation files (the "Software"), to deal
% in the Software without restriction, including without limitation the rights
% to use, copy, modify, merge, publish, distribute, sublicense, and/or sell
% copies of the Software, and to permit persons to whom the Software is
% furnished to do so, subject to the following conditions:
%
% The above copyright notice and this permission notice shall be included in
% all copies or substantial portions of the Software.
%
% THE SOFTWARE IS PROVIDED "AS IS", WITHOUT WARRANTY OF ANY KIND, EXPRESS OR
% IMPLIED, INCLUDING BUT NOT LIMITED TO THE WARRANTIES OF MERCHANTABILITY,
% FITNESS FOR A PARTICULAR PURPOSE AND NONINFRINGEMENT. IN NO EVENT SHALL THE
% AUTHORS OR COPYRIGHT HOLDERS BE LIABLE FOR ANY CLAIM, DAMAGES OR OTHER
% LIABILITY, WHETHER IN AN ACTION OF CONTRACT, TORT OR OTHERWISE, ARISING FROM,
% OUT OF OR IN CONNECTION WITH THE SOFTWARE OR THE USE OR OTHER DEALINGS IN
% THE SOFTWARE.
% -----------------------------------------------------------------------------

\def \topDirectory {../..}

\documentclass[10pt, compress]{beamer}

\usepackage{\topDirectory/template/style/directives}
% -----------------------------------------------------------------------------
% The MIT License (MIT)
%
% Copyright (c) 2015 Pejman Ghorbanzade
%
% Permission is hereby granted, free of charge, to any person obtaining a copy
% of this software and associated documentation files (the "Software"), to deal
% in the Software without restriction, including without limitation the rights
% to use, copy, modify, merge, publish, distribute, sublicense, and/or sell
% copies of the Software, and to permit persons to whom the Software is
% furnished to do so, subject to the following conditions:
%
% The above copyright notice and this permission notice shall be included in
% all copies or substantial portions of the Software.
%
% THE SOFTWARE IS PROVIDED "AS IS", WITHOUT WARRANTY OF ANY KIND, EXPRESS OR
% IMPLIED, INCLUDING BUT NOT LIMITED TO THE WARRANTIES OF MERCHANTABILITY,
% FITNESS FOR A PARTICULAR PURPOSE AND NONINFRINGEMENT. IN NO EVENT SHALL THE
% AUTHORS OR COPYRIGHT HOLDERS BE LIABLE FOR ANY CLAIM, DAMAGES OR OTHER
% LIABILITY, WHETHER IN AN ACTION OF CONTRACT, TORT OR OTHERWISE, ARISING FROM,
% OUT OF OR IN CONNECTION WITH THE SOFTWARE OR THE USE OR OTHER DEALINGS IN
% THE SOFTWARE.
% -----------------------------------------------------------------------------

\course{id}{CS114}
\course{name}{Introduction to Java}
\course{venue}{Mon/Wed, 5:30 PM - 6:45 PM}
\course{semester}{Fall 2015}
\course{department}{Department of Computer Science}
\course{university}{University of Massachusetts Boston}

\instructor{name}{Pejman Ghorbanzade}
\instructor{title}{}
\instructor{position}{Student Instructor}
\instructor{email}{pejman@cs.umb.edu}
\instructor{phone}{617-287-6419}
\instructor{office}{S-3-124B}
\instructor{office-hours}{Mon/Wed 16:00-17:30}
\instructor{address}{University of Massachusetts Boston, 100 Morrissey Blvd., Boston, MA}

\usepackage{\topDirectory/template/style/beamerthemeUmassLecture}
\doc{number}{8}
%\setbeamertemplate{footline}[text line]{}

\usepackage{wrapfig}
\usepackage{verbatimbox}

\begin{document}
\prepareCover

\section{Course Administration}

\begin{slide}
	\begin{itemize}
		\item[] Assignment 1 Due October 5, 2015 at 17:30 PM.
		\item[] Assignment 2 Released. Due October 19, 2015 at 17:30 PM.
	\end{itemize}
\end{slide}

\begin{slide}
	\begin{block}{Overview}
		\begin{itemize}
			\item[] Loops
		\end{itemize}
	\end{block}
\end{slide}

\section{Loops}

\begin{slide}
	\begin{block}{\texttt{HelloWorld.java}}
		\begin{minted}[fontsize=\small,tabsize=8, linenos, firstnumber=1]{java}
			public class HelloWorld {
			    public static void main(String[] args) {
			        System.out.println("Hello World!");
			    }
			}
		\end{minted}
	\end{block}
	\begin{block}{Objective}
		Write a program \texttt{HelloWorld100.java} so that it prints \texttt{Hello World!} a hundred times.
	\end{block}
\end{slide}

\begin{slide}
	\begin{block}{\texttt{HelloWorld100.java}}
		\begin{minted}[fontsize=\small,tabsize=8, linenos, firstnumber=1]{java}
			public class HelloWorld {
			    public static void main(String[] args) {
			        int i = 0;
			        while (i < 100) {
			            System.out.println("Hello World!");
			            i++;
			        }
			    }
			}
		\end{minted}
	\end{block}
\end{slide}

\begin{slide}
	\begin{block}{\texttt{while} Loop}
		The \texttt{while} statement continually executes a block of statements while a particular condition is \texttt{true}.
		\begin{minted}[fontsize=\small,tabsize=8]{text}
			while (condition) {
			    statement(s);
			}
		\end{minted}
	\end{block}
	\begin{block}{Remember}
		\begin{itemize}
			\item[] Condition is always evaluated first.
			\item[] Statements are executed \textbf{while} condition is evaluated as \texttt{true}.
			\item[] Condition will be reevaluated once \textbf{all} statements are executed.
		\end{itemize}
	\end{block}
\end{slide}

\begin{slide}
	\begin{block}{Objective}
		Write a program \texttt{twoSix.java} that keeps rolling two dice until you get two six.
	\end{block}
\end{slide}

\begin{slide}
	\begin{block}{\texttt{TwoSix.java} (v1.0)}
		\begin{minted}[fontsize=\small,tabsize=8, linenos, firstnumber=1]{java}
			public class TwoSix {
			    public static void main(String[] args) {
			        boolean roll = true;
			        int dice1, dice2;
			        while (roll) {
			            dice1 = (int) Math.ceil( Math.random() * 6 );
			            dice2 = (int) Math.ceil( Math.random() * 6 );
			            System.out.printf("Dice 1: %d\tDice2: %d\n",
			                dice1, dice2);
			            if (dice1 + dice2 == 12)
			                roll = false;
			        }
			    }
			}
		\end{minted}
	\end{block}
\end{slide}

\begin{slide}
	\begin{block}{\texttt{TwoSix.java} (v2.0)}
		\begin{minted}[fontsize=\small,tabsize=8, linenos, firstnumber=1]{java}
			public class TwoSix {
			    public static void main(String[] args) {
			        int dice1, dice2;
			        do {
			            dice1 = (int) Math.ceil( Math.random() * 6 );
			            dice2 = (int) Math.ceil( Math.random() * 6 );
			            System.out.printf("Dice 1: %d\tDice2: %d\n",
			                    dice1, dice2);
			        } while (dice1 + dice2 < 12);
			    }
			}
		\end{minted}
	\end{block}
\end{slide}

\begin{slide}
	\begin{block}{\texttt{do-while} Loop}
		The \texttt{do-while} loop is a variation of \texttt{while} loop which evaluates iteration condition after execution of block of statements.
		\begin{minted}[fontsize=\small,tabsize=8]{text}
			do {
			    statement(s);
			} while (condition);
		\end{minted}
	\end{block}
	\begin{block}{Remember}
		\begin{itemize}
			\item[] Condition is evaluated once \textbf{all} statements are executed.
			\item[] Statements are reexecuted \textbf{while} condition is evaluated as \texttt{true}.
			\item[] Condition will be reevaluated once \textbf{all} statements are reexecuted.
		\end{itemize}
	\end{block}
\end{slide}

\begin{slide}
	\begin{block}{\texttt{HelloWorld100.java} (v1.0)}
		\begin{minted}[fontsize=\small,tabsize=8, linenos, firstnumber=3]{java}
			int i = 0;
			while (i < 100) {
			    System.out.println("Hello World!");
			    i++;
			}
		\end{minted}
	\end{block}
	\begin{block}{Anathomy of a Code}
		\begin{minted}[fontsize=\small,tabsize=8]{text}
			intitialize iteration counter;
			while (iteration counter has not reached threshold) {
			    execute block of statements;
			    increment iteration counter;
			}
		\end{minted}
	\end{block}
\end{slide}

\begin{slide}
	\begin{block}{\texttt{HelloWorld100.java} (v2.0)}
		\begin{minted}[fontsize=\small,tabsize=8, linenos, firstnumber=1]{java}
			public class HelloWorld {
			    public static void main(String[] args) {
			        for (int i = 0; i < 100; i++)
			            System.out.println("Hello World!");
			    }
			}
		\end{minted}
	\end{block}
\end{slide}

\begin{slide}
	\begin{block}{\texttt{for} Loop}
		The \texttt{for} loop is a special case of a \texttt{while} loop that executes a block of statements for a specified number of times. Any \texttt{for} loop can be converted to a \texttt{while} loop.
		\begin{minted}[fontsize=\small,tabsize=8]{text}
			for (initialization, condition, afterthought) {
			    statement(s);
			}
		\end{minted}
		\begin{enumerate}
			\item Initialization\\Intitializes iteration counter
			\item Condition\\Specifies iteration condition
			\item Afterthought\\Increments/Decrements iteration counter
		\end{enumerate}
	\end{block}
\end{slide}

\begin{slide}
	\begin{block}{Objective}
		Write a program \texttt{Triangle.java} that takes number $n$ from user and prints a traingle in $n$ lines using star characters, such that there are $k$ star characters in $k^{th}$ line.
		\begin{verbbox}
		*
		**
		***
		****
		*****
		\end{verbbox}
		\begin{figure}[H]\centering
		\theverbbox
		\caption{Sample riangle generated for $n = 5$}
		\end{figure}
	\end{block}
\end{slide}

\begin{slide}
	\begin{block}{\texttt{Triangle.java}}
		\begin{minted}[fontsize=\small,tabsize=8, linenos, firstnumber=1]{java}
			public class Triangle {
			    public static void main(String[] args) {
			        Scanner input = new Scanner(System.in);
			        int numRow = input.nextInt();
			        input.close();
			        for (int i = 0; i < numRow; i++) {
			            for (int j = 0; j <= i; j++)
			                System.out.print("*");
			            System.out.printf("\n");
			        }
			    }
			}
		\end{minted}
	\end{block}
\end{slide}

\plain{}{Keep Calm\\and\\Practice}

\end{document}
