% -----------------------------------------------------------------------------
% The MIT License (MIT)
%
% Copyright (c) 2015 Pejman Ghorbanzade
%
% Permission is hereby granted, free of charge, to any person obtaining a copy
% of this software and associated documentation files (the "Software"), to deal
% in the Software without restriction, including without limitation the rights
% to use, copy, modify, merge, publish, distribute, sublicense, and/or sell
% copies of the Software, and to permit persons to whom the Software is
% furnished to do so, subject to the following conditions:
%
% The above copyright notice and this permission notice shall be included in
% all copies or substantial portions of the Software.
%
% THE SOFTWARE IS PROVIDED "AS IS", WITHOUT WARRANTY OF ANY KIND, EXPRESS OR
% IMPLIED, INCLUDING BUT NOT LIMITED TO THE WARRANTIES OF MERCHANTABILITY,
% FITNESS FOR A PARTICULAR PURPOSE AND NONINFRINGEMENT. IN NO EVENT SHALL THE
% AUTHORS OR COPYRIGHT HOLDERS BE LIABLE FOR ANY CLAIM, DAMAGES OR OTHER
% LIABILITY, WHETHER IN AN ACTION OF CONTRACT, TORT OR OTHERWISE, ARISING FROM,
% OUT OF OR IN CONNECTION WITH THE SOFTWARE OR THE USE OR OTHER DEALINGS IN
% THE SOFTWARE.
% -----------------------------------------------------------------------------

\def \topDirectory {../..}

\documentclass[10pt, compress]{beamer}

\usepackage{\topDirectory/template/style/directives}
% -----------------------------------------------------------------------------
% The MIT License (MIT)
%
% Copyright (c) 2015 Pejman Ghorbanzade
%
% Permission is hereby granted, free of charge, to any person obtaining a copy
% of this software and associated documentation files (the "Software"), to deal
% in the Software without restriction, including without limitation the rights
% to use, copy, modify, merge, publish, distribute, sublicense, and/or sell
% copies of the Software, and to permit persons to whom the Software is
% furnished to do so, subject to the following conditions:
%
% The above copyright notice and this permission notice shall be included in
% all copies or substantial portions of the Software.
%
% THE SOFTWARE IS PROVIDED "AS IS", WITHOUT WARRANTY OF ANY KIND, EXPRESS OR
% IMPLIED, INCLUDING BUT NOT LIMITED TO THE WARRANTIES OF MERCHANTABILITY,
% FITNESS FOR A PARTICULAR PURPOSE AND NONINFRINGEMENT. IN NO EVENT SHALL THE
% AUTHORS OR COPYRIGHT HOLDERS BE LIABLE FOR ANY CLAIM, DAMAGES OR OTHER
% LIABILITY, WHETHER IN AN ACTION OF CONTRACT, TORT OR OTHERWISE, ARISING FROM,
% OUT OF OR IN CONNECTION WITH THE SOFTWARE OR THE USE OR OTHER DEALINGS IN
% THE SOFTWARE.
% -----------------------------------------------------------------------------

\course{id}{CS114}
\course{name}{Introduction to Java}
\course{venue}{Mon/Wed, 5:30 PM - 6:45 PM}
\course{semester}{Fall 2015}
\course{department}{Department of Computer Science}
\course{university}{University of Massachusetts Boston}

\instructor{name}{Pejman Ghorbanzade}
\instructor{title}{}
\instructor{position}{Student Instructor}
\instructor{email}{pejman@cs.umb.edu}
\instructor{phone}{617-287-6419}
\instructor{office}{S-3-124B}
\instructor{office-hours}{Mon/Wed 16:00-17:30}
\instructor{address}{University of Massachusetts Boston, 100 Morrissey Blvd., Boston, MA}

\usepackage{\topDirectory/template/style/beamerthemeUmassLecture}
\doc{number}{9}
%\setbeamertemplate{footline}[text line]{}

\begin{document}
\prepareCover

\section{Course Administration}

\begin{slide}
	\begin{itemize}
		\item[] Midterm exam to be held Ocober 21, 2015 at 5:30 PM.
	\end{itemize}
\end{slide}

\begin{slide}
	\begin{block}{Overview}
		\begin{itemize}
			\item[] Branches
			\item[] One-Dimensional Arrays
		\end{itemize}
	\end{block}
\end{slide}

\section{Branches}

\begin{slide}
	\begin{block}{Objective}
		Write a program \texttt{FindP.java} that takes a String as command-line argument and checks if it contains the letter \texttt{p} (case insensitive). Do not use method \texttt{String.contains()}.
	\end{block}
\end{slide}

\begin{slide}
	\begin{block}{FindP.java (v1)}
		\begin{minted}[fontsize=\small,tabsize=8, linenos, firstnumber=1]{java}
			public class FindP {
			    public static void main(String[] args) {
			        String phrase = args[0].toLowerCase();
			        char letter = 'p';
			        boolean found = false;
			        for (int i = 0; i < phrase.length(); i++) {
			            if (phrase.charAt(i) == letter)
			                found = true;
			        }
			        if (found)
			            System.out.println("Found some '" + letter "'.");
			        else
			            System.out.println("No '"+ letter +"' found.");
			    }
			}
		\end{minted}
	\end{block}
\end{slide}

\begin{slide}
	\begin{block}{Result}
		\begin{minted}[fontsize=\small,tabsize=8]{text}
			$ java FindP Hello
			No 'p' Found.
			$ java FindP Pejman
			Found Some 'p'.
		\end{minted}
	\end{block}
	\begin{block}{Question}
		How is the for loop executed?
	\end{block}
\end{slide}

\begin{slide}
	\begin{block}{FindP.java (v2)}
		\begin{minted}[fontsize=\small,tabsize=8, linenos, firstnumber=1]{java}
			public class FindP {
			    public static void main(String[] args) {
			        String phrase = args[0].toLowerCase();
			        char letter = 'p';
			        for (int i = 0; i < phrase.length(); i++) {
			            if (phrase.charAt(i) == letter)
			                System.out.print(letter);
			            else
			                System.out.print("-");
			        }
			    }
			}
		\end{minted}
	\end{block}
\end{slide}

\begin{slide}
	\begin{block}{Result}
		\begin{minted}[fontsize=\small,tabsize=8]{text}
			$ java FindP Happy
			--pp-
			$ java FindP Pejman
			p-----
		\end{minted}
	\end{block}
	\begin{block}{Problem Statement}
		There are as many loop iterations as \texttt{phrase.length()}. Are they really required?
	\end{block}
\end{slide}

\begin{slide}
	\begin{block}{\texttt{break} Statement}
		\texttt{break} statement immediately terminates the loop.
		Control flow will return to the statement after the loop.
	\end{block}
\end{slide}

\begin{slide}
	\begin{block}{FindP.java (v3)}
		\begin{minted}[fontsize=\small,tabsize=8, linenos, firstnumber=3]{java}
			String phrase = args[0].toLowerCase();
			char letter = 'p';
			boolean found = false;
			for (int i = 0; i < phrase.length(); i++)
			    if (phrase.charAt(i) == letter) {
			        found = true;
			        break;
			    }
			if (found)
			    System.out.println("Found some '" + letter "'.");
			else
			    System.out.println("No '"+ letter +"' found.");
		\end{minted}
	\end{block}
\end{slide}

\begin{slide}
	\begin{block}{FindP.java (v4)}
		\begin{minted}[fontsize=\small,tabsize=8, linenos, firstnumber=1]{java}
			String phrase = args[0].toLowerCase();
			char letter = 'p';
			for (int i = 0; i < phrase.length(); i++) {
			    if (phrase.charAt(i) == letter) {
			        System.out.print(letter);
			        break;
			    }
			    else
			        System.out.print("-");
			}
		\end{minted}
	\end{block}
	\begin{block}{Result}
		\begin{minted}[fontsize=\small,tabsize=8]{text}
			$ java FindP Happy
			--p
			$ java FindP Pejman
			p
		\end{minted}
	\end{block}
\end{slide}

\begin{slide}
	\begin{block}{Note}
	Unconditional branches can be avoided to enhance code readability.
	\end{block}
	\begin{block}{FindP.java (v5)}
		\begin{minted}[fontsize=\small, tabsize=8, linenos, firstnumber=1]{java}
			String phrase = args[0].toLowerCase();
			char letter = 'p';
			boolean found = false;
			int i = 0;
			while (!found && i < phrase.length())
			if (phrase.charAt(i++) == letter)
			    found = true;
			    System.out.println(found ? "Found" : "Not Found");
		\end{minted}
	\end{block}
\end{slide}

\begin{slide}
	\begin{block}{\texttt{continue} Statement}
		The \texttt{continue} statement skips \emph{current iteration} of the loop.
		Control flow will immediately skip to the end the loop's body.
	\end{block}
\end{slide}

\begin{slide}
	\begin{block}{Objective}
		Write a program \texttt{CountP.java} that takes a String as command-line argument and prints number of 'p' letters it contains.
		Comparison is case insensitive.
	\end{block}
\end{slide}

\begin{slide}
	\begin{block}{CountP.java (v1)}
		\begin{minted}[fontsize=\small,tabsize=8, linenos, firstnumber=1]{java}
			public class CountP {
			    public static void main(String[] args) {
			        String phrase = args[0].toLowerCase();
			        char letter = 'p';
			        int counter = 0;
			        for (int i = 0; i < phrase.length(); i++) {
			            if (phrase.charAt(i) != letter)
			                continue;
			            counter++;
			        }
			        System.out.printf("Found %d 'P'(s)\n", counter);
			    }
			}
		\end{minted}
	\end{block}
\end{slide}

\begin{slide}
	\begin{block}{Note}
	Unconditional branches can be avoided to enhance code readability.
	\end{block}
	\begin{block}{CountP.java (v2)}
		\begin{minted}[fontsize=\small,tabsize=8, linenos, firstnumber=3]{java}
			String phrase = args[0].toLowerCase();
			char letter = 'p';
			int counter = 0;
			for (int i = 0; i < phrase.length(); i++) {
			    if (phrase.charAt(i) == letter)
			        counter++;
			}
			System.out.printf("Found %d 'P'(s)\n", counter);
		\end{minted}
	\end{block}
\end{slide}

\begin{slide}
	\begin{block}{\texttt{return} statement}
	Exits from the current method.
	Control flow returns to where the method was invoked.
	\texttt{return} statement will be discussed later.
	\end{block}
\end{slide}

\section{One-Dimensional Arrays}

\begin{slide}
	\begin{block}{Objective}
		Write a program \texttt{RandomCard.java} that randomly selects a card from a standard 52-card deck and prints its name.
	\end{block}
\end{slide}

\begin{slide}
	\begin{block}{\texttt{RandomCard.java} (v1)}
		\begin{minted}[fontsize=\small,tabsize=8, linenos, firstnumber=1]{java}
			public class RandomCard {
			    public static void main(String[] args) {
			        int numCard = 52;
			        int randNum = Math.floor(Math.random() * numCard);
			        switch (randNum) {
			            case 0: cardName = "Ace of Spades"; break;
			            case 1: cardName = "Two of Spades"; break;
			            // Seriously?
		\end{minted}
		\begin{minted}[fontsize=\small,tabsize=8, linenos, firstnumber=57]{java}
			            case 50: cardName = "Queen of Hearts"; break;
			            case 51: cardName = "King of Hearts"; break;
			        }
			    }
			}
		\end{minted}
	\end{block}
\end{slide}

\begin{slide}
	\begin{block}{Advice from a Fellow Programmer}
		Never repeat yourself!
	\end{block}
	\begin{block}{Problem Statment}
		Simple program ends up in 60 lines of code.
	\end{block}
	\begin{block}{Proposed Solution}
		Separate suits and ranks.
	\end{block}
\end{slide}

\begin{slide}
	\begin{block}{\texttt{RandomCard.java} (v2) (Page 1)}
		\begin{minted}[fontsize=\small, tabsize=8, linenos, firstnumber=1]{java}
			public class RandomCard {
			    public static void main(String[] args) {
			        int numCard = 52;
			        int randNum = Math.floor(Math.random() * numCard);
			        suitNum = randNum / 13;
			        switch (suitNum) {
			            case 0: suitName = "Spades"; break;
			            case 1: suitName = "Diamonds"; break;
			            case 2: suidName = "Clubs"; break;
			            case 3: suitName = "Hearts"; break;
			        }
		\end{minted}
	\end{block}
\end{slide}

\begin{slide}
	\begin{block}{\texttt{RandomCard.java} (v2) (Page 2)}
		\begin{minted}[fontsize=\small, tabsize=8, linenos, firstnumber=12]{java}
			        switch cardNum = numCard % 13;
			        switch (cardNum) {
			            case 0: cardName = "Ace"; break;
			            case 1: cardName = "Two"; break;
			            // Not cool!
		\end{minted}
		\begin{minted}[fontsize=\small, tabsize=8, linenos, firstnumber=26]{java}
			            case 11: cardName = "Queen"; break;
			            case 12: cardName = "King"; break;
			        }
			        System.out.println(cardName + " of " + suitName);
			    }
			}
		\end{minted}
	\end{block}
\end{slide}

\begin{slide}
	\begin{block}{Problem Statement}
		\begin{itemize}
			\item[] Access to names of values and suits are limited.
			\item[] Same code should be executed every time name of card is needed.
			\item[] Solution is not expandable.
		\end{itemize}
	\end{block}
	\begin{block}{Clues}
		\begin{itemize}
			\item[] All cases assign string literals to one single variable.
			\item[] Name of $13^{th}$ card numbers for every suit is ``King".
		\end{itemize}
	\end{block}
\end{slide}

\begin{slide}
	\begin{block}{Proposed Solution}
		Define an ordered collection of card names and get name of a card by calling its position in collection.
		\begin{minted}[fontsize=\small,tabsize=8]{java}
			String[] SuitNames = {
			    "Spades", "Diamonds", "Clubs", "Hearts"
			};
			String[] CardNames = {
			    "Ace", "Two", "Three", "Four", "Five",
			    "Six", "Seven", "Eight", "Nine", "Ten",
			    "Soldier", "Queen", "King"
			};
		\end{minted}
	\end{block}
\end{slide}

\begin{slide}
	\begin{block}{Definition}
		\begin{itemize}
			\item[] Array is a data structure that holds a \texttt{fixed} number of elements of a \texttt{single} type.
			\item[] Each array element is identified by its \texttt{numerical index}.
			\item[] An array is stored based on position of its \texttt{first} element.
			\item[] First element of an array has index 0.
		\end{itemize}
	\end{block}
\end{slide}

\begin{slide}
	\begin{block}{Sample Usage}
		\begin{minted}[fontsize=\small,tabsize=8]{java}
			int[] primes;
			primes = new int[4];
			primes[0] = 2;
			primes[1] = 3;
			primes[2] = 5;
			primes[3] = 7;
			for (int i = 0; i < 4; i++)
			    System.out.printf("Prime %d is %d\n", i + 1, primes[i]);
		\end{minted}
	\end{block}
\end{slide}

\begin{slide}
	\begin{block}{\texttt{RandomCard.java} (v3)}
		\begin{minted}[fontsize=\small, tabsize=8, linenos, firstnumber=5]{java}
			String[] SuitNames = {
			    "Spades", "Diamonds", "Clubs", "Hearts"
			};
			String[] CardNames = {
			    "Ace", "Two", "Three", "Four", "Five",
			    "Six", "Seven", "Eight", "Nine", "Ten",
			    "Soldier", "Queen", "King"
			};
			suitNum = randNum / 13;
			cardNum = randNum % 13;
			suitName = suitNames[suitNum];
			cardName = cardNames[cardNum];
			System.out.println(cardName + " of " + suitName);
		\end{minted}
	\end{block}
\end{slide}

\begin{slide}
	\begin{block}{Advantages}
		\begin{itemize}
			\item[] Data is more organized.
			\item[] Code reuse is now possible.
			\item[] Solution is expandable.
		\end{itemize}
	\end{block}
	\begin{block}{Disadvantages}
		\begin{itemize}
			\item[] Once declared, size of array is fixed.
			\item[] Insertion and deletion are not possible.
		\end{itemize}
	\end{block}
\end{slide}

\begin{slide}
	\begin{block}{Objective}
		Write a program \texttt{PrimeNumber.java} that prints all first 1000 prime numbers.
	\end{block}
\end{slide}

\begin{slide}
	\begin{block}{\texttt{PrimeNumber.java} (v1)}
		\begin{minted}[fontsize=\small,tabsize=8, linenos, firstnumber=3]{java}
			int counter = 0;
			int number = 2;
			boolean definitelyNotPrime;
			while (counter < 1000) {
			    definitelyNotPrime = false;
			    for (int i = 2; i < number / 2; i++)
			        if (number % i == 0) {
			            definitelyNotPrime = true;
			            break;
			        }
			    if (!definitelyNotPrime)
			        System.out.printf("%4d: %d\n", ++counter, number);
			    number++;
			}
		\end{minted}
	\end{block}
\end{slide}

\begin{slide}
	\begin{block}{Problem Statement}
		\begin{itemize}
			\item[] To check if number $N$ is prime, it suffices to check whether it is devisible by at least one prime number less than or equal to $\frac{N}{2}$.
			\item[] It is not necessary to check all numbers less than $\frac{N}{2}$.
			\item[] What if we are asked to print $1^{st}$, $3^{rd}$, $5^{th}$, ... prime numbers, after printing all first 1000 prime numbers?
		\end{itemize}
	\end{block}
\end{slide}

\begin{slide}
	\begin{block}{Objective}
		Write a program \texttt{PrimeNumber.java} that first prints all first 1000 prime numbers then prints $500^{th}$ prime number.
	\end{block}
\end{slide}

\begin{slide}
	\begin{block}{\texttt{PrimeNumber.java} (v2)}
		\begin{minted}[fontsize=\small, tabsize=8, linenos, firstnumber=3]{java}
			int[] primeNumbers = new int[1000];
			int counter = 0;
			int number = 2;
			boolean definitelyNotPrime;
			while (counter < primeNumbers.length) {
			    definitelyNotPrime = false;
			    for (int i = 0; i < counter && !definitelyNotPrime; i++)
			        if (number % primeNumbers[i] == 0)
			            definitelyNotPrime = true;
			    if (!definitelyNotPrime) {
			        primeNumbers[counter++] = number;
			        System.out.printf("%4d: %d\n", counter, number);
			    }
			    number++;
			}
			System.out.printf("%4d: %d\n", 500, primeNumbers[499]);
		\end{minted}
	\end{block}
\end{slide}

\begin{slide}
	\begin{block}{Remember}
		\begin{itemize}
			\item[] Array is the most useful data structure that significantly organizes your code.
			\item[] Arrays hold a \texttt{fixed} number of elements of a \texttt{single} type.
		\end{itemize}
	\end{block}
\end{slide}

\plain{}{Keep Calm\\and\\Practice}

\end{document}
