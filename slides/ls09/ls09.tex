% -----------------------------------------------------------------------------
% The MIT License (MIT)
%
% Copyright (c) 2015 Pejman Ghorbanzade
%
% Permission is hereby granted, free of charge, to any person obtaining a copy
% of this software and associated documentation files (the "Software"), to deal
% in the Software without restriction, including without limitation the rights
% to use, copy, modify, merge, publish, distribute, sublicense, and/or sell
% copies of the Software, and to permit persons to whom the Software is
% furnished to do so, subject to the following conditions:
%
% The above copyright notice and this permission notice shall be included in
% all copies or substantial portions of the Software.
%
% THE SOFTWARE IS PROVIDED "AS IS", WITHOUT WARRANTY OF ANY KIND, EXPRESS OR
% IMPLIED, INCLUDING BUT NOT LIMITED TO THE WARRANTIES OF MERCHANTABILITY,
% FITNESS FOR A PARTICULAR PURPOSE AND NONINFRINGEMENT. IN NO EVENT SHALL THE
% AUTHORS OR COPYRIGHT HOLDERS BE LIABLE FOR ANY CLAIM, DAMAGES OR OTHER
% LIABILITY, WHETHER IN AN ACTION OF CONTRACT, TORT OR OTHERWISE, ARISING FROM,
% OUT OF OR IN CONNECTION WITH THE SOFTWARE OR THE USE OR OTHER DEALINGS IN
% THE SOFTWARE.
% -----------------------------------------------------------------------------

\def \topDirectory {../..}

\documentclass[10pt, compress]{beamer}

\usepackage{\topDirectory/template/style/directives}
% -----------------------------------------------------------------------------
% The MIT License (MIT)
%
% Copyright (c) 2015 Pejman Ghorbanzade
%
% Permission is hereby granted, free of charge, to any person obtaining a copy
% of this software and associated documentation files (the "Software"), to deal
% in the Software without restriction, including without limitation the rights
% to use, copy, modify, merge, publish, distribute, sublicense, and/or sell
% copies of the Software, and to permit persons to whom the Software is
% furnished to do so, subject to the following conditions:
%
% The above copyright notice and this permission notice shall be included in
% all copies or substantial portions of the Software.
%
% THE SOFTWARE IS PROVIDED "AS IS", WITHOUT WARRANTY OF ANY KIND, EXPRESS OR
% IMPLIED, INCLUDING BUT NOT LIMITED TO THE WARRANTIES OF MERCHANTABILITY,
% FITNESS FOR A PARTICULAR PURPOSE AND NONINFRINGEMENT. IN NO EVENT SHALL THE
% AUTHORS OR COPYRIGHT HOLDERS BE LIABLE FOR ANY CLAIM, DAMAGES OR OTHER
% LIABILITY, WHETHER IN AN ACTION OF CONTRACT, TORT OR OTHERWISE, ARISING FROM,
% OUT OF OR IN CONNECTION WITH THE SOFTWARE OR THE USE OR OTHER DEALINGS IN
% THE SOFTWARE.
% -----------------------------------------------------------------------------

\course{id}{CS114}
\course{name}{Introduction to Java}
\course{venue}{Mon/Wed, 5:30 PM - 6:45 PM}
\course{semester}{Fall 2015}
\course{department}{Department of Computer Science}
\course{university}{University of Massachusetts Boston}

\instructor{name}{Pejman Ghorbanzade}
\instructor{title}{}
\instructor{position}{Student Instructor}
\instructor{email}{pejman@cs.umb.edu}
\instructor{phone}{617-287-6419}
\instructor{office}{S-3-124B}
\instructor{office-hours}{Mon/Wed 16:00-17:30}
\instructor{address}{University of Massachusetts Boston, 100 Morrissey Blvd., Boston, MA}

\usepackage{\topDirectory/template/style/beamerthemeUmassLecture}
\doc{number}{9}
%\setbeamertemplate{footline}[text line]{}

\begin{document}
\prepareCover

\section{Course Administration}

\begin{slide}
	\begin{itemize}
		\item[] Midterm exam to be held Ocober 21, 2015 at 5:30 PM.
	\end{itemize}
\end{slide}

\begin{slide}
	\begin{block}{Overview}
		\begin{itemize}
			\item[] Branches
			\item[] One-Dimensional Arrays
		\end{itemize}
	\end{block}
\end{slide}

\section{Branches}

\begin{slide}
	\begin{block}{Objective}
		Write a program \texttt{FindP.java} that takes a String as command-line argument and checks if it contains the letter \texttt{p} (case insensitive). Do not use method \texttt{String.contains()}.
	\end{block}
\end{slide}

\begin{slide}
	\begin{block}{FindP.java (v1)}
		\begin{minted}[fontsize=\small,tabsize=8, linenos, firstnumber=1]{java}
			public class FindP {
			    public static void main(String[] args) {
			        String phrase = args[0].toLowerCase();
			        char letter = 'p';
			        boolean found = false;
			        for (int i = 0; i < phrase.length(); i++) {
			            if (phrase.charAt(i) == letter)
			                found = true;
			        }
			        if (found)
			            System.out.println("Found some '" + letter "'.");
			        else
			            System.out.println("No '"+ letter +"' found.");
			    }
			}
		\end{minted}
	\end{block}
\end{slide}

\begin{slide}
	\begin{block}{Result}
		\begin{minted}[fontsize=\small,tabsize=8]{text}
			> javac FindP.java
			> java FindP Hello
			No 'p' Found.
			> java FindP Pejman
			Found Some 'p'.
		\end{minted}
	\end{block}
	\begin{block}{Question}
		How is the for loop executed?
	\end{block}
\end{slide}

\begin{slide}
	\begin{block}{FindP.java (v2)}
		\begin{minted}[fontsize=\small,tabsize=8, linenos, firstnumber=1]{java}
			public class FindP {
			    public static void main(String[] args) {
			        String phrase = args[0].toLowerCase();
			        char letter = 'p';
			        for (int i = 0; i < phrase.length(); i++) {
			            if (phrase.charAt(i) == letter)
			                System.out.print(letter);
			            else
			                System.out.print("-");
			        }
			    }
			}
		\end{minted}
	\end{block}
\end{slide}

\begin{slide}
	\begin{block}{Result}
		\begin{minted}[fontsize=\small,tabsize=8]{text}
			> javac FindP.java
			> java FindP Happy
			--pp-
			> java FindP Pejman
			p-----
		\end{minted}
	\end{block}
	\begin{block}{Problem Statement}
		There are as many loop iterations as \texttt{phrase.length()}. Are they really required?
	\end{block}
\end{slide}

\begin{slide}
	\begin{block}{\texttt{break} Statement}
		\texttt{break} statement immediately terminates the loop.
		Control flow will return to the statement after the loop.
	\end{block}
\end{slide}

\begin{slide}
	\begin{block}{FindP.java (v3)}
		\begin{minted}[fontsize=\small,tabsize=8, linenos, firstnumber=3]{java}
			String phrase = args[0].toLowerCase();
			char letter = 'p';
			boolean found = false;
			for (int i = 0; i < phrase.length(); i++)
			    if (phrase.charAt(i) == letter) {
			        found = true;
			        break;
			    }
			if (found)
			    System.out.println("Found some '" + letter "'.");
			else
			    System.out.println("No '"+ letter +"' found.");
		\end{minted}
	\end{block}
\end{slide}

\begin{slide}
	\begin{block}{FindP.java (v4)}
		\begin{minted}[fontsize=\small,tabsize=8, linenos, firstnumber=1]{java}
			String phrase = args[0].toLowerCase();
			char letter = 'p';
			for (int i = 0; i < phrase.length(); i++) {
			    if (phrase.charAt(i) == letter) {
			        System.out.print(letter);
			        break;
			    }
			    else
			        System.out.print("-");
			}
		\end{minted}
	\end{block}
	\begin{block}{Result}
		\begin{minted}[fontsize=\small,tabsize=8]{text}
			> javac FindP.java
			> java FindP Happy
			--p
		\end{minted}
	\end{block}
\end{slide}

\begin{slide}
	\begin{block}{Note}
	Unconditional branches can be avoided to enhance code readability.
	\end{block}
	\begin{block}{FindP.java (v5)}
		\begin{minted}[fontsize=\small, tabsize=8, linenos, firstnumber=1]{java}
			String phrase = args[0].toLowerCase();
			char letter = 'p';
			boolean found = false;
			int i = 0;
			while (!found && i < phrase.length())
			if (phrase.charAt(i++) == letter)
			    found = true;
			    System.out.println(found ? "Found" : "Not Found");
		\end{minted}
	\end{block}
\end{slide}

\begin{slide}
	\begin{block}{\texttt{continue} Statement}
		The \texttt{continue} statement skips \emph{current iteration} of the loop.
		Control flow will immediately skip to the end the loop's body.
	\end{block}
\end{slide}

\begin{slide}
	\begin{block}{Objective}
		Write a program \texttt{CountP.java} that takes a String as command-line argument and prints number of 'p' letters it contains. Comparison is case insensitive.
	\end{block}
\end{slide}

\begin{slide}
	\begin{block}{CountP.java (v1)}
		\begin{minted}[fontsize=\small,tabsize=8, linenos, firstnumber=1]{java}
			public class CountP {
			    public static void main(String[] args) {
			        String phrase = args[0].toLowerCase();
			        char letter = 'p';
			        int counter = 0;
			        for (int i = 0; i < phrase.length(); i++) {
			            if (phrase.charAt(i) != letter)
			                continue;
			            counter++;
			        }
			        System.out.printf("Found %d 'P'(s)\n", counter);
			    }
			}
		\end{minted}
	\end{block}
\end{slide}

\begin{slide}
	\begin{block}{Note}
	Unconditional branches can be avoided to enhance code readability.
	\end{block}
	\begin{block}{CountP.java (v2)}
		\begin{minted}[fontsize=\small,tabsize=8, linenos, firstnumber=3]{java}
			String phrase = args[0].toLowerCase();
			char letter = 'p';
			int counter = 0;
			for (int i = 0; i < phrase.length(); i++) {
			    if (phrase.charAt(i) == letter)
			        counter++;
			}
			System.out.printf("Found %d 'P'(s)\n", counter);
		\end{minted}
	\end{block}
\end{slide}

\begin{slide}
	\begin{block}{\texttt{return} statement}
	Exits from the current method. Control flow returns to where the method was invoked.

	Will be discussed later.
	\end{block}
\end{slide}

\plain{}{Keep Calm\\and\\Practice}

\end{document}
