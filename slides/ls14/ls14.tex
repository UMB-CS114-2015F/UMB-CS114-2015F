% -----------------------------------------------------------------------------
% The MIT License (MIT)
%
% Copyright (c) 2015 Pejman Ghorbanzade
%
% Permission is hereby granted, free of charge, to any person obtaining a copy
% of this software and associated documentation files (the "Software"), to deal
% in the Software without restriction, including without limitation the rights
% to use, copy, modify, merge, publish, distribute, sublicense, and/or sell
% copies of the Software, and to permit persons to whom the Software is
% furnished to do so, subject to the following conditions:
%
% The above copyright notice and this permission notice shall be included in
% all copies or substantial portions of the Software.
%
% THE SOFTWARE IS PROVIDED "AS IS", WITHOUT WARRANTY OF ANY KIND, EXPRESS OR
% IMPLIED, INCLUDING BUT NOT LIMITED TO THE WARRANTIES OF MERCHANTABILITY,
% FITNESS FOR A PARTICULAR PURPOSE AND NONINFRINGEMENT. IN NO EVENT SHALL THE
% AUTHORS OR COPYRIGHT HOLDERS BE LIABLE FOR ANY CLAIM, DAMAGES OR OTHER
% LIABILITY, WHETHER IN AN ACTION OF CONTRACT, TORT OR OTHERWISE, ARISING FROM,
% OUT OF OR IN CONNECTION WITH THE SOFTWARE OR THE USE OR OTHER DEALINGS IN
% THE SOFTWARE.
% -----------------------------------------------------------------------------

\def \topDirectory {../..}

\documentclass[10pt, compress]{beamer}

\usepackage{\topDirectory/template/style/directives}
% -----------------------------------------------------------------------------
% The MIT License (MIT)
%
% Copyright (c) 2015 Pejman Ghorbanzade
%
% Permission is hereby granted, free of charge, to any person obtaining a copy
% of this software and associated documentation files (the "Software"), to deal
% in the Software without restriction, including without limitation the rights
% to use, copy, modify, merge, publish, distribute, sublicense, and/or sell
% copies of the Software, and to permit persons to whom the Software is
% furnished to do so, subject to the following conditions:
%
% The above copyright notice and this permission notice shall be included in
% all copies or substantial portions of the Software.
%
% THE SOFTWARE IS PROVIDED "AS IS", WITHOUT WARRANTY OF ANY KIND, EXPRESS OR
% IMPLIED, INCLUDING BUT NOT LIMITED TO THE WARRANTIES OF MERCHANTABILITY,
% FITNESS FOR A PARTICULAR PURPOSE AND NONINFRINGEMENT. IN NO EVENT SHALL THE
% AUTHORS OR COPYRIGHT HOLDERS BE LIABLE FOR ANY CLAIM, DAMAGES OR OTHER
% LIABILITY, WHETHER IN AN ACTION OF CONTRACT, TORT OR OTHERWISE, ARISING FROM,
% OUT OF OR IN CONNECTION WITH THE SOFTWARE OR THE USE OR OTHER DEALINGS IN
% THE SOFTWARE.
% -----------------------------------------------------------------------------

\course{id}{CS114}
\course{name}{Introduction to Java}
\course{venue}{Mon/Wed, 5:30 PM - 6:45 PM}
\course{semester}{Fall 2015}
\course{department}{Department of Computer Science}
\course{university}{University of Massachusetts Boston}

\instructor{name}{Pejman Ghorbanzade}
\instructor{title}{}
\instructor{position}{Student Instructor}
\instructor{email}{pejman@cs.umb.edu}
\instructor{phone}{617-287-6419}
\instructor{office}{S-3-124B}
\instructor{office-hours}{Mon/Wed 16:00-17:30}
\instructor{address}{University of Massachusetts Boston, 100 Morrissey Blvd., Boston, MA}

\usepackage{\topDirectory/template/style/beamerthemeUmassLecture}
\doc{number}{14}
%\setbeamertemplate{footline}[text line]{}

\begin{document}
\prepareCover

\section{Course Administration}

\begin{slide}
	\begin{block}{Overview}
		\begin{itemize}
			\item[] Access Modifiers
			\item[] Encapsulation
		\end{itemize}
	\end{block}
\end{slide}

\section{Access Modifiers}

\begin{slide}
	\begin{block}{Definition}
		Access modifiers specify scope of classes or class members (data memebrs, methods, constructors).
		Access modifiers determine whether other classes can use a particular data member or invoke a particular method.
	\end{block}
	\begin{block}{Types}
		\begin{columns}
		\column{0.5\textwidth}
			\begin{itemize}
				\item[] public
				\item[] default (no modifier)
			\end{itemize}
		\column{0.5\textwidth}
			\begin{itemize}
			\item[] protected
			\item[] private
			\end{itemize}
		\end{columns}
	\end{block}
\end{slide}

\begin{slide}
	\begin{block}{\texttt{public} modifier}
		Classes or class members may be declared as \alert{public} in which case they are accessible to all classes everywhere.
	\end{block}
	\begin{block}{\texttt{default} modifier}
		Classes or class members may be declared with no explicit modifiers, in which case they are only accessible within their package.
	\end{block}
\end{slide}

\begin{slide}
	\begin{block}{\texttt{protected} modifier}
		Class members may be declared as \alert{protected} in which case they are accessible within their package.
	\end{block}
	\begin{block}{\texttt{private} modifier}
		Class members may be declared as \alert{private} in which case they are only accessible within their class.
	\end{block}
\end{slide}

\begin{slide}
	\begin{block}{Objective}
		Write a class \texttt{Employee.java} for an application \texttt{EmployeeTest.java} that manages salary calculations for interns and full-time employees.
		Interns are paid \$25/h for the number of hours they worked every week.
		Full-time employees are paid \$50/h for 40 hours every week.
		Each employee has a name and a unique SSN.
	\end{block}
\end{slide}

\begin{slide}
	\begin{block}{Class \texttt{Employee.java} (v1.0) (Part 1)}
		\begin{minted}[fontsize=\small,tabsize=8, linenos, firstnumber=1]{java}
			public class Employee {
			  // class variables
			  public static int normalHours = 40;
			  public static int internRate = 25;
			  public static int fullTimeRate = 50;
			  public static int numEmployees = 1;
			  // instance variables
			  public int empId;
			  public String name;
			  public boolean isIntern;
			  public double numHours = 0;
			  public double perHour;
		\end{minted}
	\end{block}
\end{slide}

\begin{slide}
	\begin{block}{Class \texttt{Employee.java} (v1.0) (Part 2)}
		\begin{minted}[fontsize=\small,tabsize=8, linenos, firstnumber=13]{java}
			  // constructor
			  public Employee(String name, boolean isIntern) {
			    this.empId = numEmployees;
			    this.name = name;
			    this.isIntern = isIntern;
			    this.perHour = isIntern ? internRate : fullTimeRate;
			    numEmployees++;
			  }
		\end{minted}
	\end{block}
\end{slide}

\begin{slide}
	\begin{block}{Class \texttt{Employee.java} (v1.0) (Part 3)}
		\begin{minted}[fontsize=\small,tabsize=8, linenos, firstnumber=21]{java}
			  //methods
			  public double getAmountToPay() {
			    if (this.isIntern) {
			      return this.numHours * this.perHour;
			    } else {
			      return normalHours * this.perHour;
			    }
			  }
			}
		\end{minted}
	\end{block}
\end{slide}

\begin{slide}
	\begin{block}{Class \texttt{EmployeeDemo.java} (v1.0)}
		\begin{minted}[fontsize=\small,tabsize=8, linenos, firstnumber=1]{java}
			public class EmployeeDemo {
			  public static void main(String[] args) {
			    Employee emp1 = new Employee("Tim", true);
			    emp1.numHours = 20;
			    Employee emp2 = new Employee("Tom", false);
			    emp1.numHours = 10;
			    System.out.println(emp1.getAmountToPay());
			    System.out.println(emp2.getAmountToPay());
			  }
			}
		\end{minted}
	\end{block}
\end{slide}

\begin{slide}
	\begin{block}{Problem Statement}
		Class and class members of \texttt{Employee.java} all have public modifiers.
		This makes it possible for everyone to access and modify data in \texttt{Employee.java} class or data of its objects, as they please.
	\end{block}
	\begin{block}{\texttt{Evil.java} (v1.0)}
		\begin{minted}[fontsize=\small,tabsize=8, linenos, firstnumber=1]{java}
			public class Evil {
			  public static void harm(Employee employee) {
			    employee.ssn = -100;
			    employee.name = "HACKED";
			    employee.perHour = -1;
			  }
			}
		\end{minted}
	\end{block}
\end{slide}

\begin{slide}
	\begin{block}{Proposed Solution}
		\begin{itemize}
			\item[] Always use as restrictive access level as possible.
			\item[] Do not allow direct access to object data.
			\item[] Provide methods as controlled ways of indirect access to data.
			\item[] Declare data that is not expected to change as constant.
		\end{itemize}
	\end{block}
\end{slide}

\begin{slide}
	\begin{block}{Class \texttt{Employee.java} (v2.0) (Part 1)}
		\begin{minted}[fontsize=\small,tabsize=8, linenos, firstnumber=1]{java}
			public class Employee {
			  // class variables
			  private static int final NORMAL_HOURS = 40;
			  private static int final RATE_INTERN = 25;
			  private static int final RATE_FULLTIME = 50;
			  private static int numEmployees = 1;
			  // instance variables
			  private int empId;
			  private String name;
			  private boolean isIntern;
			  private double numHours = 0;
		\end{minted}
	\end{block}
\end{slide}

\begin{slide}
	\begin{block}{Class \texttt{Employee.java} (v2.0) (Part 2)}
		\begin{minted}[fontsize=\small,tabsize=8, linenos, firstnumber=12]{java}
			  // constructor
			  public Employee(String name, boolean isIntern) {
			    this.empId = numEmployees;
			    this.name = name;
			    this.isIntern = isIntern;
			    numEmployees++;
			  }
		\end{minted}
	\end{block}
\end{slide}

\begin{slide}
	\begin{block}{Class \texttt{Employee.java} (v2.0) (Part 3)}
		\begin{minted}[fontsize=\small,tabsize=8, linenos, firstnumber=19]{java}
			  //methods
			  public void work(double numHours) {
			    this.numHours += numHours;
			  }
			  public void hire() {
			    this.isIntern = false;
			  }
			  public double getAmountToPay() {
			    if (this.isIntern == true) {
			      return this.numHours * RATE_INTERN;
			    } else {
			      return NORMAL_HOUR * RATE_FULLTIME;
			    }
			  }
			}
		\end{minted}
	\end{block}
\end{slide}

\begin{slide}
	\begin{block}{Class \texttt{EmployeeDemo.java} (v2.0)}
		\begin{minted}[fontsize=\small,tabsize=8, linenos, firstnumber=1]{java}
			public class EmployeeDemo {
			  public static void main(String[] args) {
			    Employee emp1 = new Employee("Tim", true);
			    emp1.work(20);
			    Employee emp2 = new Employee("Tom", false);
			    emp1.work(10);
			    System.out.println(emp1.getAmountToPay());
			    System.out.println(emp2.getAmountToPay());
			  }
			}
		\end{minted}
	\end{block}
\end{slide}

\section{Encapsulation}

\begin{slide}
	\begin{block}{Definition}
		Encapsulation promotes controlling data of an object by restricting its access to methods of its class.
		By encapsulation the states of an object, direct data manipulation is disallowed.
		Indirect access and control is selectively given only to needed states.
	\end{block}
\end{slide}

\begin{slide}
	\begin{block}{Advantages}
		\begin{description}
			\item[Modularity] Objects are passed to classes instead of data.
			\item[Data Hiding] Details of internal implementation are not disclosed.
			\item[Code Re-Use] Once verified, a developed class can be trusted.
			\item[Pluggability] Problems will be specific to classes.
		\end{description}
	\end{block}
\end{slide}

\begin{slide}
	\begin{block}{Approach}
		\begin{enumerate}
			\item[] Use private access modifier for data members.
			\item[] Develop accessors and mutators for data members.
			\item[] Impose restrictions for accessors and mutators.
			\item[] Develop additional methods for data modification if required.
		\end{enumerate}
	\end{block}
\end{slide}

\begin{slide}
	\begin{block}{Accessors}
		Accessors (getters) return some property of an object.
		\begin{minted}[fontsize=\small,tabsize=8]{java}
			  public int getEmployeeId() {
			    return this.empId;
			  }
			  public String getName() {
			    return this.name;
			  }
		\end{minted}
	\end{block}
\end{slide}

\begin{slide}
	\begin{block}{Mutators}
		Mutators (setters) change some property of an object.
		\begin{minted}[fontsize=\small,tabsize=8]{java}
			  public void setName(String name) {
			    this.name = name;
			  }
			  public void setStatus(boolean isIntern) {
			    this.isIntern = isIntern;
			  }
		\end{minted}
	\end{block}
\end{slide}

\begin{slide}
	\begin{block}{Naming}
		There is no limit in how you name your accessors and mutators.
		By convention however, we usually use \texttt{getPropertyName} and \texttt{setPropertyName} to name accessors and mutators respectively.
		For properties with \texttt{boolean} data type, \texttt{isPropertyName} is used by convention.
	\end{block}
\end{slide}

\begin{slide}
	\begin{block}{Remember}
		Accessors and mutators are often declared public and used to access the property outside the object.
	\end{block}
	\begin{block}{Immutable Properties}
		Mutators can be omitted to prohibit modifications of certain attributes.
	\end{block}
\end{slide}

\plain{}{Keep Calm\\and\\Practice}

\end{document}
