%%%%%%%%%%%%%%%%%%%%%%%%%%%%%%%%%%%%%%%%%%%%%%%%%%%%%%%%%%%%%%%%%%%%%%%%%%%%%%
% CS114: Introduction to Programming in Java
% Copyright 2015 Pejman Ghorbanzade <mail@ghorbanzade.com>
% Creative Commons Attribution-ShareAlike 4.0 International License
% https://github.com/ghorbanzade/UMB-CS114-2015F/blob/master/LICENSE
%%%%%%%%%%%%%%%%%%%%%%%%%%%%%%%%%%%%%%%%%%%%%%%%%%%%%%%%%%%%%%%%%%%%%%%%%%%%%%

\def \topDirectory {.}
\def \texDirectory {\topDirectory/src/main/tex}

\documentclass[12pt,letterpaper,twoside]{article}
\usepackage{\texDirectory/template/style/directives}
\usepackage{\texDirectory/template/style/assignment}
% -----------------------------------------------------------------------------
% The MIT License (MIT)
%
% Copyright (c) 2015 Pejman Ghorbanzade
%
% Permission is hereby granted, free of charge, to any person obtaining a copy
% of this software and associated documentation files (the "Software"), to deal
% in the Software without restriction, including without limitation the rights
% to use, copy, modify, merge, publish, distribute, sublicense, and/or sell
% copies of the Software, and to permit persons to whom the Software is
% furnished to do so, subject to the following conditions:
%
% The above copyright notice and this permission notice shall be included in
% all copies or substantial portions of the Software.
%
% THE SOFTWARE IS PROVIDED "AS IS", WITHOUT WARRANTY OF ANY KIND, EXPRESS OR
% IMPLIED, INCLUDING BUT NOT LIMITED TO THE WARRANTIES OF MERCHANTABILITY,
% FITNESS FOR A PARTICULAR PURPOSE AND NONINFRINGEMENT. IN NO EVENT SHALL THE
% AUTHORS OR COPYRIGHT HOLDERS BE LIABLE FOR ANY CLAIM, DAMAGES OR OTHER
% LIABILITY, WHETHER IN AN ACTION OF CONTRACT, TORT OR OTHERWISE, ARISING FROM,
% OUT OF OR IN CONNECTION WITH THE SOFTWARE OR THE USE OR OTHER DEALINGS IN
% THE SOFTWARE.
% -----------------------------------------------------------------------------

\course{id}{CS114}
\course{name}{Introduction to Java}
\course{venue}{Mon/Wed, 5:30 PM - 6:45 PM}
\course{semester}{Fall 2015}
\course{department}{Department of Computer Science}
\course{university}{University of Massachusetts Boston}

\instructor{name}{Pejman Ghorbanzade}
\instructor{title}{}
\instructor{position}{Student Instructor}
\instructor{email}{pejman@cs.umb.edu}
\instructor{phone}{617-287-6419}
\instructor{office}{S-3-124B}
\instructor{office-hours}{Mon/Wed 16:00-17:30}
\instructor{address}{University of Massachusetts Boston, 100 Morrissey Blvd., Boston, MA}


\usepackage{amsmath}

\begin{document}

\doc{title}{Assignment 3}
\doc{date-pub}{Oct 19, 2015 at 5:30 PM}
\doc{date-due}{Nov 04, 2015 at 5:30 PM}
\doc{points}{8}

\prepare{header}

\section*{Question 1}

Write a program \texttt{HouseOfCards.java} that randomly selects two cards from a standard 52-card deck.
The program would then print the names of the cards based on their rank and suit.
Finally, your program should determine if the two cards share the same rank or the same suit.
Your program is expected to function as shown in following examples:

\begin{terminal}
$ java HouseOfCards
Queen of Diamonds
Jack of Spades
Cards do not share rank or suit.
$ java HouseOfCards
Ace of Hearts
8 of Hearts
Cards share the same suit.
\end{terminal}

\newpage

\section*{Question 2}

Write a program \texttt{Deviation.java} that prompts user to enter a positive integer number $n$ and then asks user to enter $n$ floating-point numbers.
Your program should compute and display standard deviation of the numbers according to Equation \ref{eq1}, in the format shown below.

\begin{equation}
\text{deviation} = \sqrt{\frac{\sum_{i = 1}^{n} \left( x_i - \frac{\sum_{i = 1}^{n} x_i}{n} \right)^2 }{n - 1}}
\label{eq1}
\end{equation}

\begin{terminal}
$ javac Deviation.java
$ java Deviation
Enter count: 10
Enter 10 numbers: 1.9 2.5 3.7 2 1 6 3 4 5 2
Standard Deviation: 1.557
\end{terminal}

\section*{Question 3}

Write a program \texttt{DuplicateTester.java} that prompts user for a positive integer $x$ and creates an array with $x$ randomly generted integer elements in range 0 to 9.
Use a method with signature \texttt{public static int[] removeDuplicates(int[] array);} to remove all duplicate elements in the array.
Your program is expected to terminate by displaying the new array.
Following is a sample expected run of the program.

\begin{terminal}
$ javac DuplicateTester.java
$ java DuplicateTester
Size of Array: 10
Generated Array:
3 4 2 8 2 4 3 1 6 7
Filtered Array:
3 4 2 8 1 6 7
\end{terminal}

\newpage

\section*{Question 4}

Thomas, a CS114 student, had been asked to write a program that performs simple arithmetic operations (add and multiply) on two matrices $x$ and $y$ whose elements are randomly generated from range 0 to 9.
The program was supposed to prompt user for a number $r$ and generate two $r \times r$ matrices, $x$ and $y$.
The program would then display $x + y$ and $x \times y$ respectively, as shown in Figure \ref{terminal1}.

\renewcommand{\lstlistingname}{Figure}
\lstset{caption=Expected Output of \texttt{MatrixWorld.java}, label=terminal1}
\begin{terminal}
Size of Matrices: 3

Matrix A:
  0   3   2
  7   0   1
  1   6   5

Matrix B:
  7   8   9
  0   3   1
  8   9   6

Sum:
  7  11  11
  7   3   2
  9  15  11

Multiplication:
 16  27  15
 57  65  69
 47  71  45
\end{terminal}

\newpage

Unfortunately, Thomas gave up while developing the requested program.
He has asked that you continue development of his program without modifying the code he has written.

\lstset{caption=}
\lstset{language=Java,tabsize=4}
\begin{lstlisting}
// written by: Thomas A. Anderson
// <thomas.anderson001@umb.edu>

import java.util.Scanner;
public class MatrixWorld {
	public static void main(String[] args) {
		int size = getSize();
		int[][] matrixA = matrixInit(size);
		int[][] matrixB = matrixInit(size);
		int[][] sum = matrixAdd(matrixA, matrixB);
		int[][] multiplication = matrixMultiply(matrixA, matrixB);
		matrixDisplay(matrixA, "Matrix A");
		matrixDisplay(matrixB, "Matrix B");
		matrixDisplay(sum, "Sum");
		matrixDisplay(multiplication, "Multiplication");
	}
	public static int getSize() {
		// To Be Developed
	}
	public static int[][] matrixInit(int size) {
		// To Be Developed
	}
	public static int[][] matrixAdd(int[][] matrixA, int[][] matrixB) {
		// To Be Developed
	}
	public static int[][] matrixMultiply(int[][] matrixA, int[][] matrixB) {
		// To Be Developed
	}
	public static void matrixDisplay(int[][] matrix, String name) {
		// To Be Developed
	}
}
\end{lstlisting}

\newpage

\section*{Question 5}

Write a program \texttt{BackwardPrimes.java} that prompts user for a positive integer number $x$ and displays $x$ prime numbers whose reversal is also a prime.
Numbers 13, 31 and 157 are instances of such numbers.
Following is a sample run of your program.

\begin{terminal}
$ javac BackwardPrimes.java
$ java BackwardPrimes
Number of Primes? 15
13 17 31 37 71 73 79 97 107 113 149 157 167 179 199
\end{terminal}

\prepare{footer}

\end{document}
