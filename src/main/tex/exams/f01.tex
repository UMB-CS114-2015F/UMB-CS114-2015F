%%%%%%%%%%%%%%%%%%%%%%%%%%%%%%%%%%%%%%%%%%%%%%%%%%%%%%%%%%%%%%%%%%%%%%%%%%%%%%
% CS114: Introduction to Programming in Java
% Copyright 2015 Pejman Ghorbanzade <mail@ghorbanzade.com>
% Creative Commons Attribution-ShareAlike 4.0 International License
% https://github.com/ghorbanzade/UMB-CS114-2015F/blob/master/LICENSE
%%%%%%%%%%%%%%%%%%%%%%%%%%%%%%%%%%%%%%%%%%%%%%%%%%%%%%%%%%%%%%%%%%%%%%%%%%%%%%

\def \topDirectory {.}
\def \texDirectory {\topDirectory/src/main/tex}

\documentclass[12pt,letterpaper,twoside]{article}
\usepackage{\texDirectory/template/style/directives}
\usepackage{\texDirectory/template/style/assignment}
% -----------------------------------------------------------------------------
% The MIT License (MIT)
%
% Copyright (c) 2015 Pejman Ghorbanzade
%
% Permission is hereby granted, free of charge, to any person obtaining a copy
% of this software and associated documentation files (the "Software"), to deal
% in the Software without restriction, including without limitation the rights
% to use, copy, modify, merge, publish, distribute, sublicense, and/or sell
% copies of the Software, and to permit persons to whom the Software is
% furnished to do so, subject to the following conditions:
%
% The above copyright notice and this permission notice shall be included in
% all copies or substantial portions of the Software.
%
% THE SOFTWARE IS PROVIDED "AS IS", WITHOUT WARRANTY OF ANY KIND, EXPRESS OR
% IMPLIED, INCLUDING BUT NOT LIMITED TO THE WARRANTIES OF MERCHANTABILITY,
% FITNESS FOR A PARTICULAR PURPOSE AND NONINFRINGEMENT. IN NO EVENT SHALL THE
% AUTHORS OR COPYRIGHT HOLDERS BE LIABLE FOR ANY CLAIM, DAMAGES OR OTHER
% LIABILITY, WHETHER IN AN ACTION OF CONTRACT, TORT OR OTHERWISE, ARISING FROM,
% OUT OF OR IN CONNECTION WITH THE SOFTWARE OR THE USE OR OTHER DEALINGS IN
% THE SOFTWARE.
% -----------------------------------------------------------------------------

\course{id}{CS114}
\course{name}{Introduction to Java}
\course{venue}{Mon/Wed, 5:30 PM - 6:45 PM}
\course{semester}{Fall 2015}
\course{department}{Department of Computer Science}
\course{university}{University of Massachusetts Boston}

\instructor{name}{Pejman Ghorbanzade}
\instructor{title}{}
\instructor{position}{Student Instructor}
\instructor{email}{pejman@cs.umb.edu}
\instructor{phone}{617-287-6419}
\instructor{office}{S-3-124B}
\instructor{office-hours}{Mon/Wed 16:00-17:30}
\instructor{address}{University of Massachusetts Boston, 100 Morrissey Blvd., Boston, MA}


\usepackage[school]{\texDirectory/template/style/pgf-umlcd/pgf-umlcd}

\begin{document}

\doc{title}{Final Practice Exam}
\doc{points}{20}

\prepare{header}

\section*{Question 1}

The following code snippets either do not compile.
There are two distinct compilation problems in each code snippet.
You are expected to find and fix all errors so that given command-line argument will lead to expected output as indicated.

\begin{enumerate}[label=\textbf{(\alph*)}]

\item Execution Script \hfill Expected Output\\
\texttt{java HelloWorld Beacons} \hfill \texttt{Hello Beacons!}

\begin{lstlisting}
public class HelloWorld {
	public static void main(String args) {
		System.out.println("Hello" + args[0] "!");
	}
}
\end{lstlisting}

\item Execution Script \hfill Expected Output\\
\texttt{java Temperature 86} \hfill \texttt{30.00}

\begin{lstlisting}
public class Temperature {
	public static void main(String[] args) {
		double tempF = args[0];
		double tempC = 5 * (tempF - 32) / 9;
		System.out.printf("%.2d\n", tempC);
	}
}
\end{lstlisting}

\newpage

\item Execution Script \hfill Expected Output\\
\texttt{java Array 3} \hfill \texttt{6}

\begin{lstlisting}
public class Array {
	public static void main(String[] args) {
		sum = 0;
		int[] array = new int[5];
		for (int i = 1; i <= 5; i++) {
			array[i] = i;
			sum += i;
		}
		System.out.println(sum);
	}
}
\end{lstlisting}

\item Execution Script \hfill Expected Output\\
\texttt{java Random 5} \hfill integer $x$ where $0 \leq x < 5$

\begin{lstlisting}
public class Random {
	public static void main(String[] args) {
		int num = args[0];
		int random = getRandomNumber(num);
		System.out.println(random);
	}
	public static int getRandomNumber(num) {
		return (int) (Math.random() * num);
	}
}
\end{lstlisting}

\newpage

\item Execution Script \hfill Expected Output\\
\texttt{java CircleDemo 1} \hfill 3.14

\begin{lstlisting}
public class CircleDemo {
	public static void main(String[] args) {
		Circle circle = Circle(Double.parseDouble(args[0]));
		System.out.printf("%.2f\n", circle.getArea());
	}
}
public class Circle {
	private double radius;
	public Circle(double radius) {
		this.radius = radius;
	}
	public double getArea() {
		double area = Math.PI * Math.pow(this.radius, 2);
	}
}
\end{lstlisting}

\end{enumerate}

\section*{Question 2}

The following code snippet compiles and runs without error.

Determine the output of the following program when user gives number five as command line argument.
Formulate the output based on any integer number given as command line argument.
Support your answer by explaining how the program works.

\lstinputlisting[firstline=12]{\topDirectory/src/main/java/f01/Math.java}

\section*{Question 3}

A student has developed the following code snippet for his CS114 assignment.
The program asks for a number $n$ and prints a triangle in $n$ number of lines
using star characters as shown below.

\lstinputlisting[firstline=12]{\topDirectory/src/main/java/f01/Triangle1.java}

\begin{terminal}
$ javac Triangle1.java
$ java Triangle1
Enter number of rows: 4
*
**
***
****
\end{terminal}

Just before submitting, the student realized that the printed triangle was supposed to be right-aligned.
Since he is running out of time, he has asked for your help to fix the problem.
Rewrite the program such to produce the output as shown below.

\begin{terminal}
$ javac Triangle2.java
$ java Triangle2
Enter number of rows: 4
   *
  **
 ***
****
\end{terminal}

\section*{Question 4}

Write a static method with the following method signature that takes an integer $n$ and returns a two-dimensional array of size $n \times n$ whose elements are randomely generated integer numbers in range $[0, 10)$.

\begin{terminal}
public static int[][] generateMatrix(int size);
\end{terminal}

\section*{Question 5}

A group of students are making an object-oriented java program that can record grades of multiple students (no more than 10) and display their averages.
The team has agreed that their program will have two classes; \texttt{Grader.java} that contains the main method and \texttt{Student.java} from which student objects are instantiated.
So far the students wrote a sample \texttt{Grader.java}, shown below, to demonstrate how student objects should be used.

\lstinputlisting[firstline=12]{\topDirectory/src/main/java/f01/Grader.java}

They are now trying to develop the Student class and are asking for your help.

\begin{enumerate}
	\item Write a UML diagram for the \texttt{Student} class that can be used with \texttt{Grader.java} file.
	\item Develop the \texttt{Student.java} file that implements your UML diagram.
\end{enumerate}

\newpage

\section*{Question 6}

A software developer was assigned to develop an Employee class for a simple accounting software.
In a development meeting, he proposed the UML diagram shown in Figure \ref{fig1} for the Employee class.
Some developers, however, argued that this design is prone to major security threats and needs minor modification.
Briefly explain what you think is the issue with this design.
Rewrite the UML diagram to fix the issue.

\begin{figure}[H]
	\centering
	\begin{tikzpicture}
		\begin{class}[text width=10cm]{Employee}{0, 0}
			\attribute{+ name: String}
			\attribute{+ numHours: num}
			\attribute{+ isIntern: boolean}
			\operation{+ Employee(name: String, isIntern: boolean)}
			\operation{+ work(numHours: int): void}
			\operation{+ isIntern(): boolean}
			\operation{+ getName(): String}
			\operation{+ getSalary(): int}
		\end{class}
	\end{tikzpicture}
	\caption{UML Diagram for Employee class}\label{fig1}
\end{figure}

\end{document}
