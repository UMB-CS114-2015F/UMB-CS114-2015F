%%%%%%%%%%%%%%%%%%%%%%%%%%%%%%%%%%%%%%%%%%%%%%%%%%%%%%%%%%%%%%%%%%%%%%%%%%%%%%
% CS114: Introduction to Programming in Java
% Copyright 2015 Pejman Ghorbanzade <mail@ghorbanzade.com>
% Creative Commons Attribution-ShareAlike 4.0 International License
% https://github.com/ghorbanzade/UMB-CS114-2015F/blob/master/LICENSE
%%%%%%%%%%%%%%%%%%%%%%%%%%%%%%%%%%%%%%%%%%%%%%%%%%%%%%%%%%%%%%%%%%%%%%%%%%%%%%

\def \topDirectory {.}
\def \texDirectory {\topDirectory/src/main/tex}
\def \javaDirectory {\topDirectory/src/main/java}

\documentclass[12pt,letterpaper,twoside]{article}
\usepackage{\texDirectory/template/style/directives}
\usepackage{\texDirectory/template/style/assignment}
% -----------------------------------------------------------------------------
% The MIT License (MIT)
%
% Copyright (c) 2015 Pejman Ghorbanzade
%
% Permission is hereby granted, free of charge, to any person obtaining a copy
% of this software and associated documentation files (the "Software"), to deal
% in the Software without restriction, including without limitation the rights
% to use, copy, modify, merge, publish, distribute, sublicense, and/or sell
% copies of the Software, and to permit persons to whom the Software is
% furnished to do so, subject to the following conditions:
%
% The above copyright notice and this permission notice shall be included in
% all copies or substantial portions of the Software.
%
% THE SOFTWARE IS PROVIDED "AS IS", WITHOUT WARRANTY OF ANY KIND, EXPRESS OR
% IMPLIED, INCLUDING BUT NOT LIMITED TO THE WARRANTIES OF MERCHANTABILITY,
% FITNESS FOR A PARTICULAR PURPOSE AND NONINFRINGEMENT. IN NO EVENT SHALL THE
% AUTHORS OR COPYRIGHT HOLDERS BE LIABLE FOR ANY CLAIM, DAMAGES OR OTHER
% LIABILITY, WHETHER IN AN ACTION OF CONTRACT, TORT OR OTHERWISE, ARISING FROM,
% OUT OF OR IN CONNECTION WITH THE SOFTWARE OR THE USE OR OTHER DEALINGS IN
% THE SOFTWARE.
% -----------------------------------------------------------------------------

\course{id}{CS114}
\course{name}{Introduction to Java}
\course{venue}{Mon/Wed, 5:30 PM - 6:45 PM}
\course{semester}{Fall 2015}
\course{department}{Department of Computer Science}
\course{university}{University of Massachusetts Boston}

\instructor{name}{Pejman Ghorbanzade}
\instructor{title}{}
\instructor{position}{Student Instructor}
\instructor{email}{pejman@cs.umb.edu}
\instructor{phone}{617-287-6419}
\instructor{office}{S-3-124B}
\instructor{office-hours}{Mon/Wed 16:00-17:30}
\instructor{address}{University of Massachusetts Boston, 100 Morrissey Blvd., Boston, MA}


\usepackage[school]{\texDirectory/template/style/pgf-umlcd/pgf-umlcd}

\begin{document}

\doc{title}{Solution to Final Exam}
\doc{points}{20}

\prepare{header}

\section*{Question 1}

The following code snippets do not compile.
There are two distinct compilation problems in each code snippet.
You are expected to find and fix all errors so that given command-line argument will lead to expected output as indicated.

\begin{enumerate}[label=\textbf{(\alph*)}]

\item Execution Script \hfill Expected Output\\
\texttt{java HelloWorld Beacons} \hfill \texttt{Hello Beacons!}

\begin{lstlisting}
public class HelloWorld {
	public static void main(String[] args) {
		System.out.printf("Hello %s!", args)
	}
}
\end{lstlisting}

\item Execution Script \hfill Expected Output\\
\texttt{java AreaCalculator 2} \hfill \texttt{12.57}

\begin{lstlisting}
public class AreaCalculator {
	public static void main(String[] args) {
		double radius = Integer.parseInt(args[0]);
		area = Math.PI * Math.pow(radius, 2);
		System.out.printf("%.2f\n", area);
	}
}
\end{lstlisting}

\newpage

\item Execution Script \hfill Expected Output\\
\texttt{java Factorial 3} \hfill \texttt{6}

\begin{lstlisting}
public class Factorial {
	public static void main(String[] args) {
		int num = Integer.parseInt(args[0]);
		int sum = 1;
		for (int i: i <= num: i++) {
			sum *= i;
		}
		System.out.println(sum);
	}
}
\end{lstlisting}

\item Execution Script \hfill Expected Output\\
\texttt{java Perimeter 3 4 5} \hfill \texttt{12}

\begin{lstlisting}
public class Perimeter {
	public static void main(String[] args) {
		double[] sides = new double[3];
		for (int i = 0; i < sides.length; i++) {
			sides[i] = Double.parseDouble(args[i]);
		}
		double perimeter = new getPerimeter(sides);
		System.out.println(perimeter);
	}
	public static double getPerimeter(double sides) {
		double perimeter = 0;
		for (int i = 0; i < sides.length; i++) {
			perimeter += sides[i];
		}
		return perimeter;
	}
}
\end{lstlisting}

\newpage

\item Execution Script \hfill Expected Output\\
\texttt{java CarTest 3 4} \hfill \texttt{5.0}

\begin{lstlisting}
public class CarTest {
	public static void main(String[] args) {
		double distX = Double.parseDouble(args[0]);
		double distY = Double.parseDouble(args[1]);
		Car myCar = Car();
		myCar.move(distX, distY);
		System.out.println(myCar.getDistance());
	}
}
public class Car {
	private double posX;
	private double posY;
	public Car() {
		this.posX = 0;
		this.posY = 0;
	}
	public void move(double distX, double distY) {
		this.posX += distX;
		this.posY += distY;
	}
	private double getDistance() {
		double a = Math.pow(this.posX, 2);
		double b = Math.pow(this.posY, 2);
		return Math.sqrt(a + b);
	}
}
\end{lstlisting}

\end{enumerate}

\newpage

\subsection*{Solution}

\begin{enumerate}[label=\textbf{(\alph*)}]

\item Execution Script \hfill Expected Output\\
\texttt{java HelloWorld Beacons} \hfill \texttt{Hello Beacons!}
\lstinputlisting[firstline=12,language=java]{\javaDirectory/f02/HelloWorld.java}

\item Execution Script \hfill Expected Output\\
\texttt{java AreaCalculator 2} \hfill \texttt{12.57}
\lstinputlisting[firstline=12,language=java]{\javaDirectory/f02/AreaCalculator.java}

\item Execution Script \hfill Expected Output\\
\texttt{java Factorial 3} \hfill \texttt{6}
\lstinputlisting[firstline=12,language=java]{\javaDirectory/f02/Factorial.java}

\item Execution Script \hfill Expected Output\\
\texttt{java Perimeter 3 4 5} \hfill \texttt{12}
\lstinputlisting[firstline=12,language=java]{\javaDirectory/f02/Perimeter.java}

\item Execution Script \hfill Expected Output\\
\texttt{java CarTest 3 4} \hfill \texttt{5.0}
\lstinputlisting[firstline=12,language=java]{\javaDirectory/f02/CarTest.java}
\lstinputlisting[firstline=12,language=java]{\javaDirectory/f02/Car.java}

\end{enumerate}

\newpage

\section*{Question 2}

A student found the following program on a scrap paper in-between pages of a Java textbook she borrowed from the library.
The program compiles and runs without error.
However, since it does not have any comment or self-explanatory method names, the student is not sure what the program does.

Read the program and briefly explain its functionality.
Formulate your answer by indicating the output of the program for an arbitrary integer $x$ as user-input.

\lstinputlisting[firstline=12]{\javaDirectory/f02/Mystery.java}

\subsection*{Solution}

The program prompts for a positive integer number $n$ and checks whether it is a prime number or not.
In case $n$ is prime, the program prints the number of prime numbers less than or equal to $n$.
Otherwise, the program prints $0$.

\section*{Question 3}

The following program prompts user for a number $n$ and prints a center-aligned triangle of the given form in $n$ number of lines.

\lstinputlisting[firstline=12]{\javaDirectory/f02/Diamond1.java}

\begin{terminal}
$ javac Diamond1.java
$ java Diamond1
Enter number of rows: 3
  *
 ***
*****
\end{terminal}

\newpage

Rewrite the program such that it produces a diamond of the following form instead of a triangle.

\begin{terminal}
$ javac Diamond2.java
$ java Diamond2
Enter number of rows: 3
  *
 ***
*****
 ***
  *
\end{terminal}

\subsection*{Solution}

\lstinputlisting[firstline=12, language=java]{\javaDirectory/f02/Diamond2.java}

\section*{Question 4}

Write a static method with the following method signature that takes two integers $n$ and $m$ and returns a two-dimensional array of characters that represents a randomly-generated barcode with two symbols `*' and ` ' in $n$ rows and $m$ columns.

\begin{terminal}
public static char[][] generateBarcode(int rows, int cols);
\end{terminal}

As an example, \texttt{generateBarcode(4, 6)} may return an array that if printed on the console will look like the following.

\begin{terminal}
 ** **
** *
* ** *
 *  *
\end{terminal}

\subsection*{Solution}

The code snippet given below illustrates how to implement the method.

\lstinputlisting[firstline=12, language=java]{\javaDirectory/f02/Barcode.java}

\section*{Question 5}

Write the UML diagram for a class \texttt{Elevator.java} that can be used to make the following program compile and execute as expected.

\lstinputlisting[firstline=12]{\javaDirectory/f02/ElevatorTest.java}

\subsection*{Solution}

The UML Diagram is provided in Figure \ref{figQ5}.

\begin{figure}[H]
	\centering
	\begin{tikzpicture}
		\begin{class}[text width=9cm]{Elevator}{0, 0}
			\attribute{- floor: int}
			\attribute{- limitUp: int}
			\attribute{- limitDown: int}
			\operation{+ Elevator()}
			\operation{+ currentLevel(): void}
			\operation{+ go(level: int): void}
		\end{class}
	\end{tikzpicture}
	\caption{UML Diagram for Elavator class}\label{figQ5}
\end{figure}

\newpage

\section*{Question 6}

A group of students are developing a simple software for teaching children how to paint using a computer.

After several meetings, the team have agreed on their design strategy to support basic geometry shapes.
They are proposing that each circle instance should have a radius and a certain position on the convas. In addition, a circle may have a border and may be colored.

Your team members have already developed \texttt{Border} and \texttt{Color} classes and are expecting you to develop the \texttt{Circle} class itself.

\begin{enumerate}
	\item Write a UML diagram for the \texttt{Circle} class that is in line with the proposed design strategy.
	\item Develop the \texttt{Circle.java} file that implements your UML diagram.
\end{enumerate}

\subsection*{Solution}

\begin{enumerate}

	\item The UML Diagram is proided in Figure \ref{figQ6}.

	\begin{figure}[H]
		\centering
		\begin{tikzpicture}
			\begin{class}[text width=9cm]{Circle}{0, 0}
				\attribute{- color: Color}
				\attribute{- border: Border}
				\attribute{- radius: double}
				\attribute{- center: Point2D}
				\operation{+ Circle(center: Point2D, radius: double)}
				\operation{+ getColor(): Color}
				\operation{+ getBorder(): Border}
				\operation{+ getRadius(): double}
				\operation{+ getCenter(): Point2D}
				\operation{+ fill(color: Color): void}
				\operation{+ setBorder(border: Border): void}
				\operation{+ resize(radius: double): void}
				\operation{+ transform(center: Point2D): void}
			\end{class}
		\end{tikzpicture}
		\caption{UML Diagram for Student class}\label{figQ6}
	\end{figure}

	\newpage

	\item Following is an implementation of the Circle class.

	\lstinputlisting[firstline=12,language=java]{\javaDirectory/f02/Circle.java}

\end{enumerate}

\end{document}
