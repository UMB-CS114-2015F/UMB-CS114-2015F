%%%%%%%%%%%%%%%%%%%%%%%%%%%%%%%%%%%%%%%%%%%%%%%%%%%%%%%%%%%%%%%%%%%%%%%%%%%%%%
% CS114: Introduction to Programming in Java
% Copyright 2015 Pejman Ghorbanzade <mail@ghorbanzade.com>
% Creative Commons Attribution-ShareAlike 4.0 International License
% https://github.com/ghorbanzade/UMB-CS114-2015F/blob/master/LICENSE
%%%%%%%%%%%%%%%%%%%%%%%%%%%%%%%%%%%%%%%%%%%%%%%%%%%%%%%%%%%%%%%%%%%%%%%%%%%%%%

\def \topDirectory {.}
\def \texDirectory {\topDirectory/src/main/tex}

\documentclass[12pt,letterpaper,twoside]{article}
\usepackage{\texDirectory/template/style/directives}
\usepackage{\texDirectory/template/style/assignment}
% -----------------------------------------------------------------------------
% The MIT License (MIT)
%
% Copyright (c) 2015 Pejman Ghorbanzade
%
% Permission is hereby granted, free of charge, to any person obtaining a copy
% of this software and associated documentation files (the "Software"), to deal
% in the Software without restriction, including without limitation the rights
% to use, copy, modify, merge, publish, distribute, sublicense, and/or sell
% copies of the Software, and to permit persons to whom the Software is
% furnished to do so, subject to the following conditions:
%
% The above copyright notice and this permission notice shall be included in
% all copies or substantial portions of the Software.
%
% THE SOFTWARE IS PROVIDED "AS IS", WITHOUT WARRANTY OF ANY KIND, EXPRESS OR
% IMPLIED, INCLUDING BUT NOT LIMITED TO THE WARRANTIES OF MERCHANTABILITY,
% FITNESS FOR A PARTICULAR PURPOSE AND NONINFRINGEMENT. IN NO EVENT SHALL THE
% AUTHORS OR COPYRIGHT HOLDERS BE LIABLE FOR ANY CLAIM, DAMAGES OR OTHER
% LIABILITY, WHETHER IN AN ACTION OF CONTRACT, TORT OR OTHERWISE, ARISING FROM,
% OUT OF OR IN CONNECTION WITH THE SOFTWARE OR THE USE OR OTHER DEALINGS IN
% THE SOFTWARE.
% -----------------------------------------------------------------------------

\course{id}{CS114}
\course{name}{Introduction to Java}
\course{venue}{Mon/Wed, 5:30 PM - 6:45 PM}
\course{semester}{Fall 2015}
\course{department}{Department of Computer Science}
\course{university}{University of Massachusetts Boston}

\instructor{name}{Pejman Ghorbanzade}
\instructor{title}{}
\instructor{position}{Student Instructor}
\instructor{email}{pejman@cs.umb.edu}
\instructor{phone}{617-287-6419}
\instructor{office}{S-3-124B}
\instructor{office-hours}{Mon/Wed 16:00-17:30}
\instructor{address}{University of Massachusetts Boston, 100 Morrissey Blvd., Boston, MA}


\begin{document}

\doc{title}{Midterm Practice Exam}
\doc{points}{20}

\prepare{header}

\section*{Question 1}

The following programs have each a \textbf{single} compilation error given below. You are expected to find and fix the error so that given command-line arguments will lead to expected output as indicated.

\begin{enumerate}[label=\textbf{(\alph*)}]

\item Execution Script \hfill Expected Output\\
\texttt{java HelloWorld} \hfill \texttt{Hello World!}
\begin{lstlisting}
public class HelloWorld {
	public static void main(String[] args) {
		System.println("Hello World!");
	}
}
\end{lstlisting}

\begin{terminal}
HelloWorld.java:3: error: cannot find symbol
		System.println("Hello World!");
1 error
\end{terminal}

\newpage

\item Execution Script \hfill Expected Output\\
\texttt{java Quadratic 1 -2 -4} \hfill \texttt{-1.236 and 3.237}
\begin{lstlisting}
public class Quadratic {
	public static void main(String[] args) {
		int a = Integer.parseInt(args[0]);
		int b = Integer.parseInt(args[1]);
		int c = Integer.parseInt(args[2]);
		double discriminant = b*b - 4*a*c;
		if (double discriminant > 0) {
			double sol1 = (- b - Math.sqrt(discriminant))/(2*a);
			double sol2 = (- b + Math.sqrt(discriminant))/(2*a);
			System.out.printf("%.3f and %.3f\n", sol1, sol2);
		}
	}
}
\end{lstlisting}

\begin{terminal}
Quadratic.java:7: error: '.class' expected
		if (double discriminant > 0) {
Quadratic.java:7: error: illegal start of expression
		if (double discriminant > 0) {
Quadratic.java:7: error: ';' expected
		if (double discriminant > 0) {
Quadratic.java:7: error: illegal start of expression
		if (double discriminant > 0) {
Quadratic.java:7: error: ';' expected
		if (double discriminant > 0) {
Quadratic.java:13: error: class, interface, or enum expected
}
6 errors
\end{terminal}

\newpage


\item Execution Script \hfill Expected Output\\
\texttt{java TwoCities Texas Boston} \hfill \texttt{Boston, Texas}
\begin{lstlisting}
public class TwoCities {
	public static void main(String[] args) {
		if (args[0].compareTo(args[1]) < 0) {
			System.out.println(args[0] + ", " + args[1]);
		else
			System.out.println(args[1] + ", " + args[0]);
		}
	}
}
\end{lstlisting}

\begin{terminal}
TwoCities.java:5: error: 'else' without 'if'
	else
1 error
\end{terminal}

\item Execution Script \hfill Expected Output\\
\texttt{java ShowNum 5} \hfill \texttt{1, 2, 3, 4, 5, }

\begin{lstlisting}
public class ShowNum {
	public static void main(String[] args) {
		int limit = Integer.parseInt(args[0]);
		for (i < limit; i++) {
			System.out.printf("%d, ", i);
		}
	}
}
\end{lstlisting}

\begin{terminal}
ShowNum.java:4: error: > expected
		for (i < limit; i++) {
ShowNum.java:4: error: not a statement
		for (i < limit; i++) {
ShowNum.java:4: error: ';' expected
		for (i < limit; i++) {
3 errors
\end{terminal}

\end{enumerate}

\newpage

\section*{Question 2}

The following program is written to take $n$ as a command line argument and print the first $n$ prime numbers.
The program has no compilation error and functions as expected.
A professional Java programmer, however, argues that it is badly written and that it has major design problem.
Read the code snippet, explain how it works and why it is deemed as inefficient.
Modify the program to resolve the argued problem.

\begin{lstlisting}
public class NPrimes {
	public static void main(String[] args) {
		int count = Integer.parseInt(args[0]);
		int number = 2;
		for (int i = 0; i < count; i++) {
			boolean isPrime = false;
			do {
				boolean flag = true;
				for (int j = 2; j < number; j++) {
					if (number % j == 0)
						flag = false;
				}
				if (flag) {
					System.out.println(number);
					isPrime = true;
				}
				number++;
			}
			while (isPrime == false);
		}
	}
}
\end{lstlisting}

\newpage

\section*{Question 3}

A palindrome is a sequence of characters which reads the same backward or forward.
Write a program \texttt{Palindrome.java} that prompts user for a phrase and checks if the phrase is a palindrome or not.
For simplicity, a palindrome is both case- and whitespace-sensitive.
Following are two expected sample runs of your program.

\begin{terminal}
$ javac Palindrome.java
$ java Palindrome
Enter a String? hello olleh
Your String is a Palindrome.
$ java Palindrome
Enter a String? radar
Your String is a Palindrome.
$ java Palindrome
Enter a String? race car
Your String is not a Palindrome.
\end{terminal}

\end{document}
